\chapter{Introducción}

Este proyecto es software libre, y está liberado con la licencia \cite{gplv3}.

\section{Motivación}

\subsection{La complejidad de los automóviles actuales}

Debido a la evolución tan rápida de los sistemas de los vehículos en las últimas décadas, nos encontramos con multitud de problemas que antes no habríamos tenido que experimentar, pero también nos otorgan una gran cantidad de información acerca de los diversos módulos del vehículo. 
Esto nos dota de una mayor seguridad a la hora de poder comprobar, de un simple vistazo, si nuestro automóvil está en buen estado, pero... ¿Cómo manejar todos esos sistemas de manera simultánea y mostrárselo al conductor? 



\subsection{Los sistemas empotrados en la sociedad actual}

Los sistemas empotrados forman parte de nuestra vida diaria, aunque en la mayoría de los casos nunca llegamos a verlos en sí mismos, sino ocultos en objetos cotidianos como los microondas, las vitrocerámicas, etcétera. Basan su funcionamiento en obtención de señales mediante sensores que dan información sobre el entorno y, tras procesar esos datos en un lapso de tiempo normalmente determinado, responden mediante un actuador dándonos la funcionalidad que buscamos. 


\newpage
\section{Estructura del trabajo}

El trabajo está compuesto de los siguientes capítulos: 

\begin{itemize}
    \item \textbf{Capítulo 1} - Introducción: Capítulo actual, en el que se habla de las bases del trabajo que se va a realizar.
    \item \textbf{Capítulo 2} - Descripción del problema: En este capítulo se hablará del objetivo del trabajo, de las restricciones que se van a encontrar y de los requisitos para llevarlo a cabo.
    \item \textbf{Capítulo 3} - Antecedentes: En este apartado se hablará de la tecnología de las ECU, y de toda la información necesaria para tener una visión completa del conjunto, todo esto para poder realizar el trabajo práctico con suficientes conocimientos.
    \item \textbf{Capítulo 4} - ... [pend]
    \item \textbf{Capítulo 5} - ... [pend]
    \item \textbf{Capítulo 6} - ... [pend]
    \item \textbf{Capítulo 7} - ... [pend]
\end{itemize}
