\chapter{Descripción del problema}

\fancyhead[R]{2. Descripción del problema}

\noindent\fbox{
	\parbox{\textwidth}{
    En este capítulo se tratará el problema que se va a intentar abordar en este trabajo, junto con una descripción del mismo, y los objetivos y restricciones para su solución.
	}
}
\section{El problema}

Debido a los avances nombrados en la introducción, la complejidad que supone entender cómo pueden funcionar todos los módulos que componen los vehículos actuales puede ser el factor que evite que una multitud de personas, aun interesadas por el tema, no encuentren el punto de inicio para comenzar a investigar la tecnología. Además, existe una gran cantidad de información de estos sistemas que no se desvela por los fabricantes de vehículos, debido al espionaje industrial y los plagios de otras empresas.

Podemos resumir las cuestiones que conforman el problema en los siguientes puntos:

\begin{itemize}
    \item A medida que avanza la tecnología, los vehículos se apoyan en más sistemas informáticos y ayudas electrónicas, alejándose de la simpleza de los automóviles antiguos en estos campos.

    \item Las empresas mantienen estos sistemas con un diseño propietario, oculto al público general, por lo que no podemos siquiera observar sus métodos.

    \item A pesar de estas desventajas, los vehículos se están actualizando a pasos agigantados cada año, y comprender las partes más importantes del conjunto podría ser muy útil para el usuario medio de estos en su día a día.

    \item Este proyecto puede aportar un enfoque dinámico para un primer acercamiento de los más jóvenes al automovilismo actual.
\end{itemize}


\section{Solucion propuesta}

La solución que se propone es la creación de un sistema que emule la ECU de un vehículo aunque de manera simplificada, mediante un sistema empotrado que soporte un sistema operativo de tiempo real, con el objetivo de poder atender a todas las peticiones en un tiempo de respuesta acotado. 

Para mantener esta Unidad de Control Electrónico (ó \textit(ECU)) accesible y poder desarrollar el proyecto sin tener que hacer una gran inversión económica, utilizaremos software de código abierto así como componentes de bajo coste para la construcción.

Si la temporización nos lo permite, también se podrán implementar estos sistemas en una maqueta de un vehículo, además de visualizar los valores de los módulos en un programa en un computador mediante una interfaz de usuario.


\section{Restricciones}

\begin{itemize}
\item Se priorizará el uso de componentes asequibles y con licencia \textit{open source}, para mantener el proyecto accesible al público general.
\item No se utilizará \textit{hardware} ni \textit{software} que sea profesional en el campo del automovilismo, pues no se dispone de los recursos económicos necesarios para tal fin.
\item El proyecto tendrá un tiempo acotado, siendo el fin de este periodo la convocatoria extraordinaria, en Septiembre de 2023.

\end{itemize}


\section{Objetivos}

El objetivo general de este trabajo es la creación de un sistema que nos permita emular el funcionamiento de varios de los subsistemas de un vehículo moderno, pero si queremos definir mejor cuáles serán las metas que queremos alcanzar, lo óptimo es segmentar este objetivo en un conjunto de subobjetivos, que dividiremos en dos tipos:
\newpage

\subsection{Objetivos de Investigación}

\begin{itemize}
    \item \textbf{O-I.1} - Comprender el funcionamiento de una centralita a nivel básico.
    \item \textbf{O-I.2} - Sintetizar un conjunto de funciones que podrían ser replicadas en el proyecto.
    \item \textbf{O-I.3} - Analizar las alternativas a la hora de implementar los diversos módulos.
    \item \textbf{O-I.4} - Conocer los diferentes sistemas del vehículo y cómo interactúan con el entorno mediante sensores y actuadores.
    \item \textbf{O-I.5} - Investigar sobre los sistemas de tiempo real y sus limitaciones.
    \item \textbf{O-I.6} - Entender las directrices del código propietario y buscar alternativas de código libre para implementar el proyecto.
    \item \textbf{O-I.7} - Analizar las licencias que se le podrían atribuir al proyecto y escoger una de estas acorde a las características.
    \item \textbf{O-I.8} - Conocer el impacto de las ECU en el panorama automovilístico actual. 
\end{itemize}

\subsection{Objetivos de Diseño}
\begin{itemize}
    \item \textbf{O-D.1} - Diseñar actores y casos de uso para encontrar los límites y soluciones de los que precise el proyecto.
    \item \textbf{O-D.2} - Diseñar un esquemático para representar cómo interactúan los diferentes módulos con la ECU, así como mostrar los datos que se transmitan.
    \item \textbf{O-D.3} - Programar la ECU para la recepción, tratamiento y envío de datos.
    \item \textbf{O-D.4} - Implementar una interfaz de usuario para poder mostrar los módulos que buscamos.
    \item \textbf{O-D.4} - Recrear este sistema en una maqueta de manera física si la temporización lo permitiera.
\end{itemize}