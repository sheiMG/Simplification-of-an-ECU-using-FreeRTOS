\chapter{Descripción del problema}
\section{El problema}

Debido a los avances nombrados en la introducción, la complejidad que supone entender cómo pueden funcionar todos los módulos que componen los vehículos actuales, puesto que son sistemas demasiado intrincados para que alguien que no pertenezca al campo de estudio/trabajo de estos puedan tener más que una imagen general del conjunto. \\

Podemos resumir las cuestiones que conforman el problema en los siguientes puntos:

\begin{itemize}
    \item A medida que avanza la tecnología, los vehículos se apoyan en más sistemas informáticos y ayudas electrónicas, alejándose de la simpleza de los automóviles antiguos en estos campos.

    \item Las empresas mantienen estos sistemas con un diseño propietario, oculto al público general, por lo que no podemos siquiera observar sus métodos.

    \item A pesar de estas desventajas, los vehículos se están actualizando a pasos agigantados cada año, y comprender las partes más importantes del conjunto podría ser muy útil para el usuario medio de estos en su día a día.

    \item Este proyecto puede aportar un enfoque dinámico para un primer acercamiento de la juventud al automovilismo actual.
\end{itemize}





\section{Solucion propuesta}



\section{Restricciones}
Pasta
NO podemos abarcar todo en una maqueta etc.
\section{Objetivos}



\begin{itemize}
    \item ¿Qué sistemas se podrían \"emular\" respecto a los vehículos reales, y con qué \textit{hardware}?
    \item ¿Con qué \textit{hardware} podemos afrontar la gestión, recepción y procesamiento de información de los diversos sistemas?
    \item ¿Cómo gestionar el \textit{firmware} y el \textit{software} para que realicen las tareas necesarias?
    \item Tras obtener los datos de estos sistemas, ¿Cómo filtrar la información para mostrarla en una interfaz, y de qué tipo?
\end{itemize}

\subsection{Objetivos de Investigación y Aprendizaje}
\subsection{Objetivos de Diseño y Desarrollo}
