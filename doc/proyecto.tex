%%%%%%%%%%%%%%%%%%%%%%%%%%%%%%%%%%%%%%%%%
% Short Sectioned Assignment LaTeX Template Version 1.0 (5/5/12)
% This template has been downloaded from: http://www.LaTeXTemplates.com
% Original author:  Frits Wenneker (http://www.howtotex.com)
% License: CC BY-NC-SA 3.0 (http://creativecommons.org/licenses/by-nc-sa/3.0/)
%%%%%%%%%%%%%%%%%%%%%%%%%%%%%%%%%%%%%%%%%

% \documentclass[paper=a4, fontsize=11pt]{scrartcl} % A4 paper and 11pt font size
\documentclass[12pt, a4paper, openany]{book}
\usepackage[T1]{fontenc} % Use 8-bit encoding that has 256 glyphs
\usepackage{fourier} % Use the Adobe Utopia font for the document - comment this line to return to the LaTeX default
\usepackage[utf8]{inputenc}
\usepackage{listings} % para insertar código con formato similar al editor
\usepackage[spanish, es-tabla]{babel} % Selecciona el español para palabras introducidas automáticamente, p.ej. "septiembre" en la fecha y especifica que se use la palabra Tabla en vez de Cuadro
\usepackage{url} % ,href} %para incluir URLs e hipervínculos dentro del texto (aunque hay que instalar href)
\usepackage{graphics,graphicx, float} %para incluir imágenes y colocarlas
\usepackage[gen]{eurosym} %para incluir el símbolo del euro
\usepackage{cite} %para incluir citas del archivo <nombre>.bib
\usepackage{enumerate}
\usepackage{hyperref}
\usepackage{graphicx}
\usepackage{tabularx}
\usepackage{booktabs}
\usepackage{float}
\usepackage{makecell}
\usepackage{amsmath}
\usepackage{fontawesome}
\usepackage[table,xcdraw]{xcolor}

\hypersetup{
	colorlinks=true,	% false: boxed links; true: colored links
	linkcolor=black,	% color of internal links
	urlcolor=cyan		% color of external links
}
\usepackage{fancyhdr} % Custom headers and footers
\pagestyle{fancyplain} % Makes all pages in the document conform to the custom headers and footers
\fancyhead[L]{} % Empty left header
\fancyhead[C]{} % Empty center header
\fancyhead[R]{} % My name
\fancyfoot[L]{} % Empty left footer
\fancyfoot[C]{} % Empty center footer
\fancyfoot[R]{\thepage} % Page numbering for right footer
%\renewcommand{\headrulewidth}{0pt} % Remove header underlines
\renewcommand{\footrulewidth}{0pt} % Remove footer underlines
\setlength{\headheight}{15.0pt} % Customize the height of the header
%\renewcommand{\familydefault}{\sfdefault}
\usepackage{titlesec, blindtext, color}


\definecolor{gray75}{gray}{0.75}
\definecolor{lightpurple}{HTML}{eeddff}
\definecolor{deeppurple}{HTML}{c58cff}

\newcommand{\hsp}{\hspace{20pt}}
\titleformat{\chapter}[hang]{\Huge\bfseries}{\thechapter\hsp\textcolor{gray75}{|}\hsp}{0pt}{\Huge\bfseries}
\setcounter{secnumdepth}{4}
\usepackage[Bjornstrup]{fncychap}

\renewcommand\DOCH{%
  \settowidth{\py}{\CNoV\thechapter}
  \addtolength{\py}{-10pt}
  \fboxsep=0pt%
  \colorbox{lightpurple}{\rule{0pt}{40pt}\parbox[b]{\textwidth}{\hfill}}%
  \kern-\py\raise20pt%
  \hbox{\color{deeppurple}\CNoV\thechapter}\\%
}

\renewcommand\DOTI[1]{%
  \nointerlineskip\raggedright%
  \fboxsep=\myhi%
  \vskip-1ex%
  \colorbox{lightpurple}{\parbox[t]{\mylen}{\CTV\FmTi{#1}}}\par\nobreak%
  \vskip 40pt%
}

\renewcommand\DOTIS[1]{%
  \fboxsep=0pt
  \colorbox{lightpurple}{\rule{0pt}{40pt}\parbox[b]{\textwidth}{\hfill}}\\%
  \nointerlineskip\raggedright%
  \fboxsep=\myhi%
  \colorbox{lightpurple}{\parbox[t]{\mylen}{\CTV\FmTi{#1}}}\par\nobreak%
  \vskip 40pt%
 }


\begin{document}

	% Plantilla portada UGR
	\begin{titlepage}
\newlength{\centeroffset}
\setlength{\centeroffset}{-0.5\oddsidemargin}
\addtolength{\centeroffset}{0.5\evensidemargin}
\thispagestyle{empty}

\noindent\hspace*{\centeroffset}\begin{minipage}{\textwidth}

\centering
\includegraphics[width=0.9\textwidth]{logos/logo_ugr.jpg}\\[1.4cm]

\textsc{ \Large TRABAJO FIN DE GRADO\\[0.2cm]}
\textsc{ GRADO EN INGENIERIA INFORMATICA}\\[1cm]

\center{\Huge\bfseries Control de elementos  }
\center{\Huge\bfseries de un automóvil mediante}
\center{\Huge\bfseries  un sistema empotrado} 
\noindent\rule[-1ex]{\textwidth}{1pt}\\[3.5ex]

\end{minipage}

\vspace{2.5cm}
\noindent\hspace*{\centeroffset}
\begin{minipage}{\textwidth}
\centering

\textbf{Autor}\\ {Sheila Martínez Gómez}\\[2.5ex]
\textbf{Director}\\ {Jesús González Peñalver}\\[2cm]
\includegraphics[width=0.3\textwidth]{logos/etsiit_logo.png}\\[0.1cm]
\textsc{Escuela Técnica Superior de Ingenierías Informática y de Telecomunicación}\\

\end{minipage}
\end{titlepage}


	% Plantilla prefacio UGR
	\thispagestyle{empty}

\begin{center}
{\large\bfseries Control de elementos de un automóvil \\ mediante un Sistema Empotrado }\\
\end{center}
\begin{center}
Sheila Martínez Gómez\\
\end{center}

%\vspace{0.7cm}

\vspace{0.5cm}
\noindent\textbf{Palabras clave}: \textit{software libre, sistema empotrado, ECU, microcontrolador, centralita, RTOS, automoción}
\vspace{0.7cm}

\noindent\textbf{Resumen}\\

El grueso de este Trabajo de Fin de Grado consiste en la programación de un sistema empotrado para controlar diversas funciones que realiza un vehículo, utilizando un RTOS (un sistema operativo de tiempo real) para garantizar la respuesta acotada a las señales enviadas por el usuario.\newline

La realización del trabajo comienza con una investigación acerca de los diferentes subsistemas que conforman un vehículo, así como su evolución e impacto en la sociedad actual. Una vez adquirida la base de conocimientos necesarios, se procede a realizar la programación de la centralita basada en el Sistema Empotrado, además de una interfaz de usuario que permita iniciar y detener las tareas asociadas a las funciones del automóvil.\newline

Finalmente se construye una maqueta de un vehículo mostrando los sistemas desarrollados durante el proyecto en funcionamiento.\newline

\cleardoublepage

\begin{center}
	{\large\bfseries Same, but in English}\\
\end{center}
\begin{center}
	Student's name\\
\end{center}
\vspace{0.5cm}
\noindent\textbf{Keywords}: \textit{open source, embedded system, ECU, microcontroller, RTOS, automotive}, \textit{floss}
\vspace{0.7cm}

\noindent\textbf{Abstract}\\

The aim of this Final Degree Project is the programming of an embedded system to control various functions performed by a vehicle, using a RTOS (real-time operating system) to guarantee a time-delimited response to the signals provided by the user.\newline

The work starts with an investigation of the different subsystems that make up a vehicle, as well as their evolution and impact on today's society. Once the required knowledge base has been acquired, the programming of the ECU (Electronic Control Unit) will be carried out, as well as the user interface to start and stop the task associated with the car's functions.\newline

Finally, a car model is built to show the functioning of the developed subsystems during the project.\newline

\cleardoublepage

\thispagestyle{empty}

\noindent\rule[-1ex]{\textwidth}{2pt}\\[4.5ex]

D. \textbf{Jesús González Peñalver}, Profesor(a) del Departamento de \textbf{Ingeniería de Computadores, Automática y Robótica}

\vspace{0.5cm}

\textbf{Informo:}

\vspace{0.5cm}

Que el presente trabajo, titulado \textit{\textbf{Control de elementos de un automóvil mediante un Sistema Empotrado}}, ha sido realizado bajo mi supervisión por \textbf{Sheila Martínez Gómez}, y autorizo la defensa de dicho trabajo ante el tribunal que corresponda.

\vspace{0.5cm}

Y para que conste, expiden y firman el presente informe en Granada a Junio de 2018.

\vspace{1cm}

\textbf{El/la director(a)/es: }

\vspace{5cm}

\noindent \textbf{Jesús González Peñalver}

\chapter*{Agradecimientos}

A mis padres y mi familia, por apoyarme aun estando tan lejos de casa, dándome su ayuda y sus ánimos en los momentos más duros de la carrera para seguir adelante.\newline

A Kike, por estar siempre a mi lado para sacarme una sonrisa, ayudarme con todo lo que está en su mano, y hacer que los días sean más bonitos y llevaderos.\newline

A todos los amigos de la carrera, y los que están lejos, a Aroa, por tener siempre alguna frase perfecta para el momento correcto, a Álex, por su comprensión y su humor, a David, por todas esas tardes en la universidad matando el tiempo (y estudiando cuando tocaba).\newline

Y a mi tutor, Jesús, por la comunicación necesaria y los consejos para llevar este trabajo de la mejor manera posible.\newline




	% Índice de contenidos
	\newpage
	\tableofcontents

	% Índice de imágenes y tablas
	\newpage
	\listoffigures

	% Si hay suficientes se incluirá dicho índice
	\listoftables 
	\newpage

	% Introducción 
	

\chapter{Introducción}

\fancyhead[R]{1. Introducción}

\noindent\fbox{
	\parbox{\textwidth}{
    En este capítulo se tratará la motivación para haber escogido este trabajo, el estado actual de la tecnología a investigar, y la estructura que tendrá esta memoria. 
	}
}
Este proyecto es software libre, y está liberado con la licencia \cite{gplv3}.


\section{Motivación}

La idea para el tema de este trabajo surgió de dos aficiones que tengo desde muy pequeña: los coches, y la informática. Siempre he querido comprender la arquitectura de los vehículos y, si bien este proyecto es una muestra muy pequeña, siento que me va a encaminar a seguir por esta rama durante mi carrera profesional. Además, el hecho de trabajar temas que no se han profundizado durante el grado me parece un reto necesario y, por qué no decirlo, divertido.

\subsection{La complejidad de los automóviles actuales}

Debido a la evolución tan rápida de los sistemas de los vehículos en las últimas décadas, nos encontramos con multitud de problemas que antes no habríamos tenido que experimentar. Actualmente, incluso un mecanismo tan sencillo como un elevalunas puede fallar con relativa facilidad, así como también los sistemas de infoentretenimiento, la dirección asistida, y mil funciones más, que anteriormente se manejaban únicamente por cables, ahora tienen varias placas, y un firmware asociado. 

Aunque estos sistemas también nos permiten obtener una gran cantidad de información acerca de los diversos módulos del vehículo, lo que nos otorga de una mayor seguridad y nos permite comprobar, de un simple vistazo, si nuestro automóvil está en buen estado, pero... ¿Cómo manejar todos esos sistemas de manera simultánea y mostrárselo al conductor? 



\subsection{Los sistemas empotrados en la sociedad actual}

Los sistemas empotrados forman parte de nuestra vida diaria, aunque en la mayoría de los casos nunca llegamos a verlos en sí mismos, sino que permanecen ocultos en objetos cotidianos como los microondas, las vitrocerámicas, etcétera. Basan su funcionamiento en la recepción de señales mediante sensores que obtienen información sobre el entorno y, tras procesar esos datos en un lapso de tiempo normalmente determinado, responden mediante un actuador (si fuera necesario) dándonos la funcionalidad que buscamos. Actualmente sería impensable prescindir de los \textit{embedded systems}, ya que resultan la forma más eficiente de construir sistemas informáticos dedicados, de tamaño reducido y adaptados a las características que requiera el objeto a controlar.


\newpage
\section{Estructura del trabajo}

El trabajo está compuesto de los siguientes capítulos: 

\begin{itemize}
    \item \textbf{Capítulo 1} - Introducción: Capítulo actual, en el que se habla de las bases del trabajo que se va a realizar.
    \item \textbf{Capítulo 2} - Descripción del problema: En este capítulo se hablará del objetivo del trabajo, de las restricciones que se van a encontrar y de los requisitos para llevarlo a cabo.
    \item \textbf{Capítulo 3} - Antecedentes: En este capítulo se hablará de la tecnología de las ECU, de su historia, y de aquellas tecnologías que sean relevantes para el proyecto. 
    \item \textbf{Capítulo 4} - Estudio de requisitos: El grueso de este capítulo será la descripción de casos de uso, requisitos y actores que definen y delimitan el alcance del proyecto 
    \item \textbf{Capítulo 5} -Análisis del problema: En el capítulo de análisis se tratarán los distintos problemas que se abordarán durante el desarrollo, véase las decisiones a la hora de escoger un hardware y un software determinado.
    \item \textbf{Capítulo 6} -  Planificación: En este capítulo se realizará una temporización para el proyecto, así como también otros factores importantes tales como el presupuesto y las fases de los sistemas a implementar. 
    \item \textbf{Capítulo 7} - Implementación: Se detallará en este capítulo todo el proceso de implementación del proyecto, diferenciado por las fases definidas en el capítulo de planificación. 
    \item \textbf{Capítulo 8} - Conclusiones y trabajos futuros - Este capítulo formará el cierre de la memoria, con valoración personal, análisis de lo que ha ido bien y lo que no, y trabajos que se podrán implementar en un futuro. 
\end{itemize}


	% Descripción del problema y hasta donde se llega
	\chapter{Descripción del problema}

\fancyhead[R]{2. Descripción del problema}

\noindent\fbox{
	\parbox{\textwidth}{
    En este capítulo se tratará el problema que se va a intentar abordar en este trabajo, junto con una descripción del mismo, y los objetivos y restricciones para su solución.
	}
}
\section{El problema}

Debido a los avances nombrados en la introducción, la complejidad que supone entender cómo pueden funcionar todos los módulos que componen los vehículos actuales puede ser el factor que evite que una multitud de personas, aun interesadas por el tema, no encuentren el punto de inicio para comenzar a investigar la tecnología. Además, existe una gran cantidad de información de estos sistemas que no se desvela por los fabricantes de vehículos, debido al espionaje industrial y los plagios de otras empresas.

Podemos resumir las cuestiones que conforman el problema en los siguientes puntos:

\begin{itemize}
    \item A medida que avanza la tecnología, los vehículos se apoyan en más sistemas informáticos y ayudas electrónicas, alejándose de la simpleza de los automóviles antiguos en estos campos.

    \item Las empresas mantienen estos sistemas con un diseño propietario, oculto al público general, por lo que no podemos siquiera observar sus métodos.

    \item A pesar de estas desventajas, los vehículos se están actualizando a pasos agigantados cada año, y comprender las partes más importantes del conjunto podría ser muy útil para el usuario medio de estos en su día a día.

    \item Este proyecto puede aportar un enfoque dinámico para un primer acercamiento de los más jóvenes al automovilismo actual.
\end{itemize}


\section{Solucion propuesta}

La solución que se propone es la creación de un sistema que emule la ECU de un vehículo aunque de manera simplificada, mediante un sistema empotrado que soporte un sistema operativo de tiempo real, con el objetivo de poder atender a todas las peticiones en un tiempo de respuesta acotado. 

Para mantener esta Unidad de Control Electrónico (ó \textit(ECU)) accesible y poder desarrollar el proyecto sin tener que hacer una gran inversión económica, utilizaremos software de código abierto así como componentes de bajo coste para la construcción.

Si la temporización nos lo permite, también se podrán implementar estos sistemas en una maqueta de un vehículo, además de visualizar los valores de los módulos en un programa en un computador mediante una interfaz de usuario.


\section{Restricciones}

\begin{itemize}
\item Se priorizará el uso de componentes asequibles y con licencia \textit{open source}, para mantener el proyecto accesible al público general.
\item No se utilizará \textit{hardware} ni \textit{software} que sea profesional en el campo del automovilismo, pues no se dispone de los recursos económicos necesarios para tal fin.
\item El proyecto tendrá un tiempo acotado, siendo el fin de este periodo la convocatoria extraordinaria, en Septiembre de 2023.

\end{itemize}


\section{Objetivos}

El objetivo general de este trabajo es la creación de un sistema que nos permita emular el funcionamiento de varios de los subsistemas de un vehículo moderno, pero si queremos definir mejor cuáles serán las metas que queremos alcanzar, lo óptimo es segmentar este objetivo en un conjunto de subobjetivos, que dividiremos en dos tipos:
\newpage

\subsection{Objetivos de Investigación}

\begin{itemize}
    \item \textbf{O-I.1} - Comprender el funcionamiento de una centralita a nivel básico.
    \item \textbf{O-I.2} - Sintetizar un conjunto de funciones que podrían ser replicadas en el proyecto.
    \item \textbf{O-I.3} - Analizar las alternativas a la hora de implementar los diversos módulos.
    \item \textbf{O-I.4} - Conocer los diferentes sistemas del vehículo y cómo interactúan con el entorno mediante sensores y actuadores.
    \item \textbf{O-I.5} - Investigar sobre los sistemas de tiempo real y sus limitaciones.
    \item \textbf{O-I.6} - Entender las directrices del código propietario y buscar alternativas de código libre para implementar el proyecto.
    \item \textbf{O-I.7} - Analizar las licencias que se le podrían atribuir al proyecto y escoger una de estas acorde a las características.
    \item \textbf{O-I.8} - Conocer el impacto de las ECU en el panorama automovilístico actual. 
\end{itemize}

\subsection{Objetivos de Diseño}
\begin{itemize}
    \item \textbf{O-D.1} - Diseñar actores y casos de uso para encontrar los límites y soluciones de los que precise el proyecto.
    \item \textbf{O-D.2} - Diseñar un esquemático para representar cómo interactúan los diferentes módulos con la ECU, así como mostrar los datos que se transmitan.
    \item \textbf{O-D.3} - Programar la ECU para la recepción, tratamiento y envío de datos.
    \item \textbf{O-D.4} - Implementar una interfaz de usuario para poder mostrar los módulos que buscamos.
    \item \textbf{O-D.4} - Recrear este sistema en una maqueta de manera física si la temporización lo permitiera.
\end{itemize}


	\chapter{Antecedentes}


\section{¿Qué es una ECU?}


Una ECU \textit{Electronic Control Unit} es un sistema empotrado, que consta de un microcontrolador especializado en la automoción. [1] (embitel.com) Permite, junto con el software y los protocolos de comunicación, y el conjunto de sensores y actuadores, controlar los sistemas eléctricos y subsistemas en un vehículo para su correcto funcionamiento. 
Esencialmente se encarga de recibir la información que le aportan los sensores acerca del entorno, procesar esa información para completar diversas tareas y, en muchos casos, enviar las directrices que se requieren a los actuadores de los componentes.\newline


\begin{figure}[h]
    \centering
    \includegraphics[width=0.5\textwidth]{imagenes/ECU_autotechdrive.png}
    \caption{Imagen de una ECU y su impacto en un vehículo. Extraído de: []}
\end{figure}


\subsection{Evolución e historia de la gestión de datos los vehículos}

Las siglas ECU no siempre han tenido el mismo significado. Inicialmente, cuando se comenzaron a utilizar (en torno a los años setenta), era para hablar de la unidad de control del motor \textit{\textbf{Engine Control Unit}}. Podemos desgajar los grandes cambios en lo siguiente[4]:

\begin{itemize}

    \item \textbf{Años 70} - Inicialmente eran dispositivos extremadamente simples, que solamente controlaban un par de solenoides en el \textbf{carburador}, el encargado de preparar la mezcla de aire y combustible en motores de gasolina, de manera que el vehículo pudiera obtener la máxima potencia de salida de la manera más eficiente.

    \begin{figure}[h]
        \centering
        \includegraphics[width=0.5\textwidth]{imagenes/esquema_carburador.png}
        \caption{Imagen simplificada de un carburador sus partes. Extraído de: [5]}
    \end{figure}
 
    \item \textbf{Años 80} - En esta época se comenzaron a utilizar los \textbf{reguladores de presión de combustible}, un componente que busca controlar y mantener constante la presión del combustible. Permitió a los fabricantes de vehículos tener mayor control a la hora de no dañar otros componentes, tales como inyectores o los conductos del sistema en general [5]. En este punto, la unidad de control del motor ya era la principal responsable de los sistemas de combustión. También comenzaron a utilizar el \textbf{Control de Lambda} para modificar parámetros de la misma, y alcanzar una mayor eficiencia.

    \begin{figure}[h]
        \centering
        \includegraphics[width=0.6\textwidth]{imagenes/esquema_rpc.png}
        \caption{Imagen simplificada de un regulador de combustible y sus partes. Extraído de: [7]}
    \end{figure}


    \item \textbf{Años 90} - Las ECUs añadieron un conjunto de medidas para mejorar la seguridad en el vehículo, siendo el \textbf{ABS ó Antiblockiersystem} la más relevante. Esta tecnología, aún en uso en los vehículos actuales, se encarga de variar la fuerza de frenado al detectar mediante sensores de revoluciones en las ruedas, evitando así que estas se bloqueen y perdamos el control del vehículo. El uso de las ECUs se extiende hacia los motores diésel, teniendo un papel esencial en el desarrollo de los vehículos turbodiésel.

    \begin{figure}[h]
        \centering
        \includegraphics[width=0.5\textwidth]{imagenes/esquema_abs.png}
        \caption{Imagen simplificada del ABS y sus partes. Extraído de: [8]}
    \end{figure}
    
 
    \item \textbf{Años 2000} - Las unidades de control comienzan a incluir la tecnología \textit{\textbf{drive-by-wire}}, un formalismo para definir la sustitución de controles mecánicos tradicionales por sistemas electrónicos [9], así como también se añade control del turbo y sistemas para minimizar emisiones y cumplir con los protocolos pertinentes.
       

    \begin{figure}[h]
        \centering
        \includegraphics[width=0.5\textwidth]{imagenes/esquema_dbw.png}
        \caption{Conceptualización del sistema \textit{drive-by-wire}. Extraído de: [10]}
    \end{figure}
          

    \item \textbf{Años 2010-Actualidad} - La centralita del motor controla todo el sistema de mezcla del combustible, emisiones, refrigeración y sistemas de aceleración del vehículo. Dejan de ser dispositivos simples, ahora tienen cientos de entradas, decenas de sensores, y forman parte de un conjunto de ECUs (ya utilizando la acepción actual del término: \textit{Electronic Control Unit}), cada una focalizada en un conjunto de tareas, yendo desde el propio control del motor, hasta los sistemas de infoentretenimiento del vehículo. 
       

    \begin{figure}[h]
        \centering
        \includegraphics[width=0.5\textwidth]{imagenes/ECU_autotechdrive_completa.png}
        \caption{Representación de las distintas ECUs que controlan un vehículo. Extraído de: [11]}
    \end{figure}
\end{itemize}

\newpage
       
\begin{figure}[h]
    \centering
    \includegraphics[width=0.54\textwidth]{imagenes/timeline_ECU.png}
    \caption{Linea temporal de las ECUs. Elaboración propia}
\end{figure}


\newpage


\subsection{Tipos de ECUs}
Como se ha visto en el anterior apartado, el significado de las siglas ECU ha variado con el tiempo, refiriéndose ahora a cada uno de los sistemas que controlan diversas partes del vehículo. A grandes rasgos, podemos hablar de los siguientes tipos[12]
\begin{itemize}
    \item \textbf{BCM - \textit{Body Control Module}}: Esta unidad controla tareas de índole variada, desde las luces, hasta el mecanismo de las ventanas, los retrovisores (si son motorizados), limpiaparabrisas, etc.
    \item \textbf{CCM - \textit{Climate Control Module}}: Este módulo se encarga de gestionar toda la climatización, temperatura, potencia de la bomba del aire acondicionado y calefacción.
    \item \textbf{ECM - \textit{Engine Control Module}}: Conocido antiguamente como ECU. Se encarga de gestionar todas las tareas del motor, ya mencionadas en el apartado anterior.
    \item \textbf{EBCM - \textit{Electronic Brake Control Module}}: Si el vehículo tiene un freno electrónico, este módulo es el encargado de su correcto funcionamiento.
    \item \textbf{ICM - \textit{Infotainment Control Module}}: Esta ECU controla todo lo relacionado con el infoentretenimiento, mayormente centrado en la tableta integrada del vehículo.
    \item \textbf{PSM - \textit{Power Steering Module}}: Este módulo gestiona las tareas que requiere la dirección para funcionar. Tiene una gran importancia, debido al uso de la dirección asistida en los vehículos actuales.
    \item \textbf{PCM - \textit{Powertrain Control Module}}: Esta ECU se encarga de la interconexión e interacción de los componentes que forman el \textbf{tren motriz}, el sistema que permite que la energía generada se transforme en movimiento sobre el terreno. Gestiona lo relacionado con el motor, la transmisión, los ejes, los diferenciales y la dirección del vehículo. 
    \item \textbf{SCM - \textit{Suspension Control Module}}: El objetivo de este módulo es controlar los sistemas de suspensión del vehículo.
    \item \textbf{TCM - \textit{Transmission Control Module}}: Este módulo controla todo lo relacionado con la transmisión, las marchas, el cambio y la entrega de potencia en las ruedas.
\end{itemize}
\newpage
\section{Diseño hardware de las centralitas}

Para tener un funcionamiento acorde a su importancia, las ECUs están constituidas por varios componentes, que serán los que se estudiarán en este apartado.

\begin{figure}[h]
    \centering
    \includegraphics[width=0.70\textwidth]{imagenes/ECU_hardware.png}
    \caption{Componentes hardware de una ECU. Elaboración propia}
\end{figure}

\subsection{Procesador central}
El microcontrolador es la piedra angular de una centralita. Su tarea varía según el tipo de ECU que lo contenga, pero generalmente consiste en realizar todo el procesamiento de datos, así como hacer uso de las lineas de comunicación para recibir y enviar información a los subsistemas requeridos. \newline
Estos microcontroladores pueden interaccionar entre sí, aun perteneciendo a diferentes centralitas, por medio de \textbf{multiplexación}, así como también de los buses o protocolos de comunicación. Esta relación permite a un sistema valerse de otros para realizar una tarea, véase el siguiente ejemplo: 

\begin{enumerate}
    \item El usuario nota que hace demasiado calor en el vehículo, por lo que decide cambiar la temperatura en la tableta central del vehículo.
    \item Este sistema, controlado por el \textbf{ICM} (\textit{Infotainment Control Module}), recibe la entrada del usuario y envía una señal a la ECU que se encarga de la climatización (\textbf{CCM}).
    \item El módulo de climatización recibe la entrada y envía una señal a sus actuadores para encender el aire acondicionado.
\end{enumerate}


Respecto a la arquitectura del microcontrolador, existen diversas opciones, todas (o casi todas) ellas oscilando entre los 8 bits y los 32 bits, siendo estos últimos los más extendidos con el \textbf{Infineon Tri-core Microcontroller[19]}. Este micro se encarga de gestionar todas las tareas de emisiones y sistemas de combustión. 

La elección entre un chip u otro también se hace en base a las necesidades del sistema, puesto que algunos, como el tren de potencia y el control del vehículo requieren de un menor tiempo de respuesta y un mayor rendimiento, mientras que el infoentretenimiento no tiene tanta restricción al ser un \textbf{sistema de tiempo real blando} (incumplir el límite acotado del tiempo de respuesta no ocasiona daños personales o materiales)

\subsection{Memoria}

En una ECU, normalmente existen tres tipos de memoria, \textbf{ROM}, \textbf{RAM} y \textbf{PROM/EEPROM}. Cada una tiene un objetivo específico y unas restricciones asociadas:

\begin{itemize}
    \item \textbf{Memoria no volátil}: La ECU almacena en este tipo de memoria información que no se debe borrar una vez se apague el vehículo, como los datos del sistema y los del usuario (configuraciones, perfiles de conducción, etcétera.).
    \begin{itemize}
        \item \textbf{ROM}: Almacena información de programación que solo puede leer la ECU. Es de solo lectura, y su existencia es de vital importancia para el funcionamiento del vehículo.
        \item \textbf{PROM/EEPROM}: Este tipo de memoria está hecha por el fabricante, con el objetivo de ajustarse a la aplicación de la transmisión, motor y demás piezas fundamentales del vehículo. Mientras que la PROM no se puede reprogramar, la EEPROM (\textit{Electrically Erasable Programmable Read-Only Memory}) permite al usuario o al operario reprogramarla con un dispositivo especial. 
    \end{itemize}
    \item \textbf{Memoria volátil}: La ECU utiliza esta memoria para almacenar datos temporales, no mantiene la información entre reinicios.
    \begin{itemize}
        \item \textbf{RAM}: Esta memoria permite almacenar datos temporales de las entradas, códigos de error para el diagnóstico, y resultados de cómputos para trabajar con los actuadores. Tiene un tiempo de acceso mucho menor a las \textbf{NVM} (\textit{Non-Volatile Memory}).
    \end{itemize}
\end{itemize}

\newpage
\subsection{Sensores y actuadores}

\begin{figure}[h]
    \centering
    \includegraphics[width=0.70\textwidth]{imagenes/diagrama_sensor_actuador.png}
    \caption{Representación de sensores y actuadores. Elaboración propia}
\end{figure}

\subsubsection*{Sensores}

Los sensores tienen un papel esencial en el diseño hardware de las ECU, siendo los que proveen al microcontrolador de entradas para llevar a cabo operaciones y hacer funcionar el sistema. Normalmente un sensor de un vehículo da una medida de voltaje, que es representada por un código en el procesador. Si este voltaje es incorrecto, el micro lo tomará como una entrada inválida y dará un código de error. 

Existen decenas de sensores en un vehículo, pero aquí se van a citar algunos de los más relevantes [22]:


\begin{itemize}
    \item \textbf{Sensor Lambda: }También conocido como el sensor de oxígeno. Se encarga de medir la cantidad de oxígeno en los gases del escape para comprobar si las emisiones del vehículo están acorde al protocolo. 
    \item \textbf{Sensor de posición del acelerador (\textit{throttle})}: Este sensor permite determinar la posición del pedal del acelerador. Si está más presionado en vehículos de combustión, enviará más aire y combustible al motor. En el caso de los eléctricos, se enviará una señal para que se aumente el voltaje.
    \item \textbf{Sensor de líquido refrigerante: }El nombre es explicativo. Otorga los datos a la ECU para controlar si el nivel está bajo y necesita encenderse un testigo para comunicárselo al conductor.
    \item \textbf{Sensor de presión de los neumáticos}: Si bien no es algo común, el uso de estos sensores de presión en los tapones para las ruedas permite dar información al sistema sobre si la presión es correcta o existe alguna anomalía.
\end{itemize}

\subsubsection*{Actuadores}

Un actuador es un dispositivo eléctrico o mecánico que, al recibir una señal eléctrica enviada por el microcontrolador, la convierte a una acción mecánica. Son la contraparte directa de los sensores. Su funcionamiento depende del tipo de actuador y la energía que requiera, pero los pasos generales del proceso son los siguientes:

\begin{enumerate}
    \item 1. \textbf{Recepción de señal:} En primer lugar la ECU envía una señal al actuador, indicándole cuál es la acción que debe realizar. 
    \item 2. \textbf{Conversión señal-energía:} El actuador convierte la señal recibida al valor equivalente en la energía que precise, ya sea mecánica, neumática o hidráulica. 
    \item 3. \textbf{Activación:} El actuador realiza la acción acorde a los valores que ha recibido de la ECU, siendo algunos casos, por ejemplo, la activación del motor que baja las ventanas, el encendido de las luces de largo alcance, o cualquier otro mecanismo. 
    \item 4. \textbf{Feedback:} Mayormente en los casos más sensibles, el actuador envía a la ECU una señal indicando que su tarea ha sido realizada.
\end{enumerate}

\subsection{Interfaces de comunicación}
Con el aumento de sensores, actuadores y subsistemas en los vehículos, se ha hecho imperativo tener interconexión entre todas estas partes. En algunos casos, una conexión lenta o ineficiente puede conllevar un grave peligro, incluso vidas en riesgo. 

Para poder subsanar este problema se han desarrollado varios tipos de interfaces de comunicación, que permiten transmitir los datos de un subsistema al microcontrolador, o a otro subsistema. Dependiendo del contexto en el que se utilice, existen varios tipos de interfaces:

\begin{itemize}
    \item \textbf{CAN} \textit({Controller Area Network}): Sistema de buses de alta integridad para la interconexión de dispositivos inteligentes[23].  Está ampliamente extendida en la automoción, y funciona mediante una tecnología \textit{peer-to-peer} en la que no existe un maestro que controle el resto de nodos. Cada nodo CAN, una vez está listo para transmitir, comprueba si el bus está libre y escribe en la red. Tiene un sistema de prioridades dependiendo del nodo que envíe o reciba la señal. 
    \item \textbf{LIN} \textit({Local Interface Network}): Subsistema de la línea CAN. Se encarga de liberar la carga del bus que se genera al tener que supervisar la línea. Permite reducir costes y optimizar la comunicación.
    \item \textbf{FlexRay}: Esta interfaz actúa entre ECUs, y está destinada a aplicaciones críticas, como la ya nombrada anteriormente \textit{drive-by-wire}. Posee una alta transmisión de datos, redundacia y tolerancia de errores. Fue lanzada en 2007 en un BMW X5, y supuso un gran avance en la seguridad de los vehículos.
\end{itemize}

\section{La ECU en los vehículos: El mejor amigo del conductor}

El ser humano ha hecho grandes avances en la tecnología, y esto no podría ser menos en los vehículos. Las ECU que, como se ha visto en el apartado anteror, inicialmente no tenían tanta importancia, ahora se han convertido en algo fundamental para nosotros. Estas centralitas, que antiguamente servían únicamente para gestionar la inyección de combustible en el el motor, ahora tienen multitud de utilidades para cada una de las partes del vehículo.

Gracias a estos sistemas, podemos obtener desde una mayor eficiencia en el uso de la fuente de energía (ya sea combustible fósil o electricidad), un control más específico de los mapas motor para mayor potencia, recolección de datos relevantes del vehículo, e incluso algunas mejoras \textit{Quality of Life (QoL)} que nos permitan una conducción más cómoda. 

Muchos de estos sistemas no solamente inciden en la eficiencia y comodidad, sino también en la seguridad del conductor y de los pasajeros abordo. Algunas de las medidas más relevantes, que vamos a categorizar entre activas y pasivas, son las siguientes[11]:

\begin{enumerate}
    \item \textbf{Seguridad activa} - Conjunto de medidas que proporcionan mayor estabilidad al vehículo, buscando evitar el accidente a priori.
    \begin{itemize}
        \item \textit{Luces adaptativas}: Esta tecnología se basa en el uso de una matriz de LEDs en las luces delanteras, al contrario de la tradicional bombilla incandescente usada anteriormente. Los beneficios de este diseño son múltiples, pues permite modificar el haz de luz para evitar posibles deslumbramientos a otros usuarios de la vía, así como también adaptarse a las condiciones de la carretera, e iluminar más eficazmente el trazado.
     
        \item \textit{Control de crucero adaptativo (ACC)}: Permite mantener una velocidad designada por el conductor, variando cuando sea necesario por las condiciones del tráfico. Esta tecnología supone una gran ayuda en el caso de que exista una distracción del conductor. Puede funcionar junto a un sistema de detección de señales para no sobrepasar la velocidad máxima permitida en esa vía.
    
        \item \textit{Sistema de alerta de tráfico y evasión de colisión (TCAS)}: Este sistema tiene como tarea principal la detección de eventos en el tráfico que puedan causar un accidente, reaccionando (normalmente accionando de manera automática el freno), o advirtiendo al conductor para que sea él el que realice la acción pertinente para evitarlo.

        \item \textit{Sistema de detección de cansancio}: Algunos fabricantes [14] comienzan a implementar esta tecnología que, mediante sensores en la posición del conductor, o analizando su conducción, permite detectar si existen síntomas de cansancio o distracción. Normalmente muestran un aviso que se repite de manera continuada hasta que se realiza una parada. 
\end{itemize}

\begin{figure}[h]
    \centering
    \includegraphics[width=0.7\textwidth]{imagenes/adaptive_lights.png}
    \caption{Comparación de la iluminación en los vehículos tras los años. Extraído de: [14]}
\end{figure}


    \item \textbf{Seguridad pasiva} - Conjunto de medidas que actúan una vez ha sucedido el accidente o el mal funcionamiento. Tienen el objetivo de minimizar los daños.

    \begin{itemize}
        \item \textit{Sistema de sensor de ocupación}: Permite detectar qué asientos están siendo utilizados, y comprobar el peso para saber si el ocupante es un niño o un adulto. Normalmente se utiliza para accionar el airbag si fuera necesario, y la altura a la que hacerlo.

        \item \textit{Notificación de colisión automática}: Este sistema envía un aviso a los servicios de emergencia de su localicación para que sea posible una intervención rápida.[16]
    \end{itemize}
\end{enumerate}




\section{La necesidad de un RTOS en las ECUs}

Como se ha mencionado en los apartados anteriores, los vehículos actuales poseen una gran multitud de sistemas y subsistemas que deben trabajar simultáneamente, teniendo unos márgenes temporales muy escasos y, en algunas situaciones, críticos para el funcionamiento del automóvil. 

Un sistema operativo que no fuera un RTOS podría causar desincronizaciones, respuestas demasiado lentas, y generar situaciones de peligro. 

Los RTOS permiten planificar, controlar y distribuir los núcleos de la CPU, para que todas esas funciones se realicen simultáneamente y en el tiempo de respuesta acotado, debido a su comportamiento determinista y predectible. 

Además, el uso de este tipo de SOs permite aislar los errores y prevenir que se extiendan al conjunto del vehículo, pues la planificación de los núcleos hace que sea posible mantener una tarea que monitorice los errores de manera continua y minimice el impacto de estos.
	% Estado del arte
	% 	1. Crítica al estado del arte
	% 	2. Propuesta	
	\chapter{Estudio de requisitos}
En este capítulo vamos a sentar las directrices del proyecto, utilizando la ingeniería de requisitos. Esto nos servirá de base para la implementación del sistema, además de marcar los límites de este y sus funcionalidades. 

\section{Actores}
Lo primero que se debe hacer a la hora de diseñar un sistema es encontrar las entidades con las que este se comunica, llamadas \textbf{actores}. 
En este caso existen dos: el conductor, que controla el vehículo, y la ECU, que maneja los subsistemas. Desde un punto de vista formal, podemos describirlos así: 


\begin{table}[H]
    \begin{tabular}{|llllll|}
    \hline
    \multicolumn{1}{|l|}{\textbf{Actor}}           & \multicolumn{4}{l|}{Conductor}                                                                                                                                                  & AC-01                 \\ \hline
    \multicolumn{1}{|l|}{\textbf{Descripción}}     & \multicolumn{5}{l|}{Es el conductor del vehículo.}                                                                                                                                                      \\ \hline
    \multicolumn{1}{|l|}{\textbf{Características}} & \multicolumn{5}{l|}{\begin{tabular}[c]{@{}l@{}}Tiene que mantener su atención centrada en la carretera, por lo que no\\  debe distraerse demasiado con complicaciones en los subsistemas.\end{tabular}} \\ \hline
    \multicolumn{1}{|l|}{\textbf{Relaciones}}      & \multicolumn{5}{l|}{Necesita la ECU para poder controlar el vehículo.}                                                                                                                                  \\ \hline
    \multicolumn{1}{|l|}{\textbf{Referencias}}     & \multicolumn{5}{l|}{CU-01 .. CU-08}                                                                                                                                                                     \\ \hline
    \multicolumn{1}{|l|}{\textbf{Autor}}           & \multicolumn{1}{l|}{Sheila Martínez}        & \multicolumn{1}{l|}{\textbf{Fecha}}        & \multicolumn{1}{l|}{25/06/2023}        & \multicolumn{1}{l|}{\textbf{Versión}}       & 1.0                   \\ \hline
    \multicolumn{6}{|l|}{\cellcolor[HTML]{DAE8FC}\textbf{Atributos}}                                                                                                                                                                                         \\ \hline
    \multicolumn{1}{|l|}{\textbf{Nombre}}          & \multicolumn{4}{l|}{\textbf{Descripción}}                                                                                                                                       & \textbf{Tipo}         \\ \hline
    \multicolumn{1}{|l|}{Alias}                    & \multicolumn{4}{l|}{\begin{tabular}[c]{@{}l@{}}Nombre que introduce el conductor \\ para ser identificado en el vehículo\end{tabular}}                                          & cadena de texto       \\ \hline
    \end{tabular}
    \end{table}

\begin{table}[H]
    \begin{tabular}{|llllll|}
    \hline
    \multicolumn{1}{|l|}{\textbf{Actor}}           & \multicolumn{4}{l|}{ECU}                                                                                                                                                                      & AC-02                   \\ \hline
    \multicolumn{1}{|l|}{\textbf{Descripción}}     & \multicolumn{5}{l|}{Es la unidad de control del vehículo.}                                                                                                                                                              \\ \hline
    \multicolumn{1}{|l|}{\textbf{Características}} & \multicolumn{5}{l|}{\begin{tabular}[c]{@{}l@{}}Controla todos los subsistemas del vehículo. Siempre debe tener \\ información actualizada de los valores de los sensores, y actuar\\ cuando sea necesario\end{tabular}} \\ \hline
    \multicolumn{1}{|l|}{\textbf{Relaciones}}      & \multicolumn{5}{l|}{Es controlada indirectamente por el conductor del vehículo}                                                                                                                                         \\ \hline
    \multicolumn{1}{|l|}{\textbf{Referencias}}     & \multicolumn{5}{l|}{CU-01 .. CU-17}                                                                                                                                                                                     \\ \hline
    \multicolumn{1}{|l|}{\textbf{Autor}}           & \multicolumn{1}{l|}{Sheila Martínez}            & \multicolumn{1}{l|}{\textbf{Fecha}}           & \multicolumn{1}{l|}{25/06/2023}           & \multicolumn{1}{l|}{\textbf{Versión}}           & 1.0                     \\ \hline
    \multicolumn{6}{|l|}{\cellcolor[HTML]{DAE8FC}\textbf{Atributos}}                                                                                                                                                                                                         \\ \hline
    \multicolumn{1}{|l|}{\textbf{Nombre}}          & \multicolumn{4}{l|}{\textbf{Descripción}}                                                                                                                                                     & \textbf{Tipo}           \\ \hline
    \multicolumn{1}{|l|}{}                         & \multicolumn{4}{l|}{}                                                                                                                                                                         &                         \\ \hline
    \end{tabular}
    \end{table}



    
\section{Casos de Uso}


Antes de comenzar a desarrollar el sistema, y una vez hemos descrito los actores, debemos asegurarnos de cuáles serán sus funcionalidades y sus características. Comenzaremos generando casos de uso, es decir, acciones (incluyendo variantes y/o errores), que un sistema puede realizar al interactuar con los actores. 

\begin{table}[H]
    \resizebox{\textwidth}{!}{%
    \begin{tabular}{|l|l|l|l|l|l|}
    \hline
    \textbf{Caso de Uso} & \multicolumn{3}{l|}{Encender motor} & \multicolumn{2}{l|}{CU-01} \\ \hline
    \textbf{Actores} & \multicolumn{5}{l|}{Conductor (I), ECU} \\ \hline
    \textbf{Tipo} & \multicolumn{5}{l|}{Primario, Esencial} \\ \hline
    \textbf{Referencias} & RF1 & \multicolumn{4}{l|}{CU-10} \\ \hline
    \textbf{Precondición} & \multicolumn{5}{l|}{El motor debe estar apagado} \\ \hline
    \textbf{Postcondición} & \multicolumn{5}{l|}{El motor se habrá encendido} \\ \hline
    \textbf{Autor} & Sheila Martínez & \textbf{Fecha} & 23/06/2023 & \textbf{Versión} & v.1 \\ \hline
    \multicolumn{6}{|l|}{\cellcolor[HTML]{ECF4FF}Propósito} \\ \hline
    \multicolumn{6}{|l|}{El conductor pulsa un botón, la ECU enciende el motor} \\ \hline
    \multicolumn{6}{|l|}{\cellcolor[HTML]{ECF4FF}Resumen} \\ \hline
    \multicolumn{6}{|l|}{\begin{tabular}[c]{@{}l@{}}1. El conductor pulsa el botón de arranque.\\  2. La ECU comprueba que el motor está apagado. \\ 3. La ECU arranca el motor.\end{tabular}} \\ \hline
    \end{tabular}%
    }
    \end{table}

    
\begin{table}[H]
    \resizebox{\textwidth}{!}{%
    \begin{tabular}{|l|l|l|l|l|l|}
    \hline
    \textbf{Caso de Uso} & \multicolumn{3}{l|}{Apagar motor} & \multicolumn{2}{l|}{CU-02} \\ \hline
    \textbf{Actores} & \multicolumn{5}{l|}{Conductor (I), ECU} \\ \hline
    \textbf{Tipo} & \multicolumn{5}{l|}{Primario, Esencial} \\ \hline
    \textbf{Referencias} & RF2 & \multicolumn{4}{l|}{CU-10} \\ \hline
    \textbf{Precondición} & \multicolumn{5}{l|}{El motor debe estar encendido} \\ \hline
    \textbf{Postcondición} & \multicolumn{5}{l|}{El motor se habrá apagado} \\ \hline
    \textbf{Autor} & Sheila Martínez & \textbf{Fecha} & 23/06/2023 & \textbf{Versión} & v.1 \\ \hline
    \multicolumn{6}{|l|}{\cellcolor[HTML]{ECF4FF}Propósito} \\ \hline
    \multicolumn{6}{|l|}{El conductor pulsa un botón, la ECU apaga el motor} \\ \hline
    \multicolumn{6}{|l|}{\cellcolor[HTML]{ECF4FF}Resumen} \\ \hline
    \multicolumn{6}{|l|}{\begin{tabular}[c]{@{}l@{}}1. El conductor pulsa el botón de apagado.\\  2. La ECU comprueba que el motor está encendido. \\ 3. La ECU apaga el motor.\end{tabular}} \\ \hline
    \end{tabular}%
    }
    \end{table}    


\begin{table}[H]
    \resizebox{\textwidth}{!}{%
    \begin{tabular}{|l|l|l|l|l|l|}
    \hline
    \textbf{Caso de Uso} & \multicolumn{3}{l|}{Encender luces} & \multicolumn{2}{l|}{CU-03} \\ \hline
    \textbf{Actores} & \multicolumn{5}{l|}{Conductor (I), ECU} \\ \hline
    \textbf{Tipo} & \multicolumn{5}{l|}{Primario, Esencial} \\ \hline
    \textbf{Referencias} & RF3 & \multicolumn{4}{l|}{CU-11} \\ \hline
    \textbf{Precondición} & \multicolumn{5}{l|}{Las luces deben estar apagadas} \\ \hline
    \textbf{Postcondición} & \multicolumn{5}{l|}{Las luces se habrán encendido} \\ \hline
    \textbf{Autor} & Sheila Martínez & \textbf{Fecha} & 23/06/2023 & \textbf{Versión} & v.1 \\ \hline
    \multicolumn{6}{|l|}{\cellcolor[HTML]{ECF4FF}Propósito} \\ \hline
    \multicolumn{6}{|l|}{El conductor pulsa un botón, la ECU enciende las luces} \\ \hline
    \multicolumn{6}{|l|}{\cellcolor[HTML]{ECF4FF}Resumen} \\ \hline
    \multicolumn{6}{|l|}{\begin{tabular}[c]{@{}l@{}}1. El conductor pulsa el botón de encendido de luces.\\  2. La ECU comprueba que las luces están apagadas. \\ 3. La ECU enciende las luces \\ 4. La ECU enciende el piloto que indica que las luces están encendidas.\end{tabular}} \\ \hline
    \end{tabular}%
    }
\end{table}


\begin{table}[H]
    \resizebox{\textwidth}{!}{%
    \begin{tabular}{|l|l|l|l|l|l|}
    \hline
    \textbf{Caso de Uso} & \multicolumn{3}{l|}{Apagar luces} & \multicolumn{2}{l|}{CU-04} \\ \hline
    \textbf{Actores} & \multicolumn{5}{l|}{Conductor (I), ECU} \\ \hline
    \textbf{Tipo} & \multicolumn{5}{l|}{Primario, Esencial} \\ \hline
    \textbf{Referencias} & RF3 & \multicolumn{4}{l|}{CU-11} \\ \hline
    \textbf{Precondición} & \multicolumn{5}{l|}{Las luces deben estar encendidas} \\ \hline
    \textbf{Postcondición} & \multicolumn{5}{l|}{Las luces se habrán apagado} \\ \hline
    \textbf{Autor} & Sheila Martínez & \textbf{Fecha} & 23/06/2023 & \textbf{Versión} & v.1 \\ \hline
    \multicolumn{6}{|l|}{\cellcolor[HTML]{ECF4FF}Propósito} \\ \hline
    \multicolumn{6}{|l|}{El conductor pulsa un botón, la ECU apaga las luces} \\ \hline
    \multicolumn{6}{|l|}{\cellcolor[HTML]{ECF4FF}Resumen} \\ \hline
    \multicolumn{6}{|l|}{\begin{tabular}[c]{@{}l@{}}1. El conductor pulsa el botón de apagado de luces.\\ 2. La ECU comprueba que las luces están encendidas.\\ 3. La ECU apaga las luces.\end{tabular}} \\ \hline
    \end{tabular}%
    }
    \end{table}

\begin{table}[H]
    \resizebox{\textwidth}{!}{%
    \begin{tabular}{|l|l|l|l|l|l|}
    \hline
    \textbf{Caso de Uso} & \multicolumn{3}{l|}{Encender intermitente} & \multicolumn{2}{l|}{CU-05} \\ \hline
    \textbf{Actores} & \multicolumn{5}{l|}{Conductor (I), ECU} \\ \hline
    \textbf{Tipo} & \multicolumn{5}{l|}{Primario, Esencial} \\ \hline
    \textbf{Referencias} & RF5 & \multicolumn{4}{l|}{CU-11} \\ \hline
    \textbf{Precondición} & \multicolumn{5}{l|}{Los intermitentes deben estar apagados} \\ \hline
    \textbf{Postcondición} & \multicolumn{5}{l|}{Se habrá accionado el intermitente de la dirección deseada} \\ \hline
    \textbf{Autor} & Sheila Martínez & \textbf{Fecha} & 23/06/2023 & \textbf{Versión} & v.1 \\ \hline
    \multicolumn{6}{|l|}{\cellcolor[HTML]{ECF4FF}Propósito} \\ \hline
    \multicolumn{6}{|l|}{El conductor acciona una palanca, la ECU enciende la luz intermitente} \\ \hline
    \multicolumn{6}{|l|}{\cellcolor[HTML]{ECF4FF}Resumen} \\ \hline
    \multicolumn{6}{|l|}{\begin{tabular}[c]{@{}l@{}}1. El conductor acciona la palanca hacia el lado deseado\\ 2. La ECU comprueba que el intermitente no está accionado\\ 3. La ECU enciende el intermitente de ese lado\end{tabular}} \\ \hline
    \end{tabular}%
    }
    \end{table}

\begin{table}[H]
    \resizebox{\textwidth}{!}{%
    \begin{tabular}{|l|l|l|l|l|l|}
    \hline
    \textbf{Caso de Uso} & \multicolumn{3}{l|}{Apagar intermitente} & \multicolumn{2}{l|}{CU-06} \\ \hline
    \textbf{Actores} & \multicolumn{5}{l|}{Conductor (I), ECU} \\ \hline
    \textbf{Tipo} & \multicolumn{5}{l|}{Primario, Esencial} \\ \hline
    \textbf{Referencias} & RF6 & \multicolumn{4}{l|}{CU-11} \\ \hline
    \textbf{Precondición} & \multicolumn{5}{l|}{El intermitente debe estar accionado hacia uno de los dos lados} \\ \hline
    \textbf{Postcondición} & \multicolumn{5}{l|}{Se habrán apagado los intermitentes} \\ \hline
    \textbf{Autor} & Sheila Martínez & \textbf{Fecha} & 23/06/2023 & \textbf{Versión} & v.1 \\ \hline
    \multicolumn{6}{|l|}{\cellcolor[HTML]{ECF4FF}Propósito} \\ \hline
    \multicolumn{6}{|l|}{El conductor devuelve una palanca a su estado inicial, la ECU apaga la luz intermitente} \\ \hline
    \multicolumn{6}{|l|}{\cellcolor[HTML]{ECF4FF}Resumen} \\ \hline
    \multicolumn{6}{|l|}{\begin{tabular}[c]{@{}l@{}}1. El conductor devuelve la palanca hacia su estado inicial\\ 2. La ECU apaga los intermitentes\end{tabular}} \\ \hline
    \end{tabular}%
    }
\end{table}

\begin{table}[H]
    \resizebox{\textwidth}{!}{%
    \begin{tabular}{|l|l|l|l|l|l|}
    \hline
    \textbf{Caso de Uso} & \multicolumn{3}{l|}{Cambiar estado luz freno} & \multicolumn{2}{l|}{CU-07} \\ \hline
    \textbf{Actores} & \multicolumn{5}{l|}{Conductor (I), ECU} \\ \hline
    \textbf{Tipo} & \multicolumn{5}{l|}{Primario, Esencial} \\ \hline
    \textbf{Referencias} & RF7 & \multicolumn{4}{l|}{CU-11} \\ \hline
    \textbf{Precondición} & \multicolumn{5}{l|}{} \\ \hline
    \textbf{Postcondición} & \multicolumn{5}{l|}{Se habrá encendido la luz de freno} \\ \hline
    \textbf{Autor} & Sheila Martínez & \textbf{Fecha} & 23/06/2023 & \textbf{Versión} & v.1 \\ \hline
    \multicolumn{6}{|l|}{\cellcolor[HTML]{ECF4FF}Propósito} \\ \hline
    \multicolumn{6}{|l|}{El conductor pisa el pedal de freno, se enciende la luz de freno} \\ \hline
    \multicolumn{6}{|l|}{\cellcolor[HTML]{ECF4FF}Resumen} \\ \hline
    \multicolumn{6}{|l|}{\begin{tabular}[c]{@{}l@{}}1. El conductor pisa el pedal de freno\\ 2. La ECU detecta que el freno ha sido pulsado\\ 3. La ECU enciende la luz de freno\\ 3.1 El conductor suelta el pedal de freno\\ 3.2 La ECU detecta que el freno ha sido soltado\\ 3.3 La ECU apaga la luz de freno\end{tabular}} \\ \hline
    \end{tabular}%
    }
\end{table}


\begin{table}[H]
    \resizebox{\textwidth}{!}{%
    \begin{tabular}{|l|l|l|l|l|l|}
    \hline
    \textbf{Caso de Uso} & \multicolumn{3}{l|}{Cambiar velocidad} & \multicolumn{2}{l|}{CU-08} \\ \hline
    \textbf{Actores} & \multicolumn{5}{l|}{Conductor (I), ECU} \\ \hline
    \textbf{Tipo} & \multicolumn{5}{l|}{Primario, Esencial} \\ \hline
    \textbf{Referencias} & RF8 & \multicolumn{4}{l|}{-} \\ \hline
    \textbf{Precondición} & \multicolumn{5}{l|}{El motor debe estar encendido} \\ \hline
    \textbf{Postcondición} & \multicolumn{5}{l|}{Se habrá variado la velocidad} \\ \hline
    \textbf{Autor} & Sheila Martínez & \textbf{Fecha} & 23/06/2023 & \textbf{Versión} & v.1 \\ \hline
    \multicolumn{6}{|l|}{\cellcolor[HTML]{ECF4FF}Propósito} \\ \hline
    \multicolumn{6}{|l|}{El conductor pisa el pedal de acelerador, el motor acelera proporcionalmente} \\ \hline
    \multicolumn{6}{|l|}{\cellcolor[HTML]{ECF4FF}Resumen} \\ \hline
    \multicolumn{6}{|l|}{\begin{tabular}[c]{@{}l@{}}1. El conductor pisa el acelerador\\ 2. La ECU detecta que ha sido pulsado y con cuánta intensidad.\\ 3. La ECU varía la velocidad en función de la intensidad.\end{tabular}} \\ \hline
    \end{tabular}%
    }
    \end{table}


\begin{table}[H]
    \resizebox{\textwidth}{!}{%
    \begin{tabular}{|l|l|l|l|l|l|}
    \hline
    \textbf{Caso de Uso} & \multicolumn{3}{l|}{Comprobar temperatura motor} & \multicolumn{2}{l|}{CU-09} \\ \hline
    \textbf{Actores} & \multicolumn{5}{l|}{ECU} \\ \hline
    \textbf{Tipo} & \multicolumn{5}{l|}{Primario, Esencial} \\ \hline
    \textbf{Referencias} & RF9 & \multicolumn{4}{l|}{CU-17} \\ \hline
    \textbf{Precondición} & \multicolumn{5}{l|}{} \\ \hline
    \textbf{Postcondición} & \multicolumn{5}{l|}{Se habrá leído y comprobado el valor de la temperatura del motor} \\ \hline
    \textbf{Autor} & Sheila Martínez & \textbf{Fecha} & 23/06/2023 & \textbf{Versión} & v.1 \\ \hline
    \multicolumn{6}{|l|}{\cellcolor[HTML]{ECF4FF}Propósito} \\ \hline
    \multicolumn{6}{|l|}{La ECU obtiene el valor de la temperatura del motor y comprueba si es correcta.} \\ \hline
    \multicolumn{6}{|l|}{\cellcolor[HTML]{ECF4FF}Resumen} \\ \hline
    \multicolumn{6}{|l|}{\begin{tabular}[c]{@{}l@{}}1. La ECU obtiene el valor de la temperatura del motor.\\ 2. Comprueba si el valor es correcto.\\ \\ 2.1 Si es incorrecto, enciende un piloto que lo indica.\end{tabular}} \\ \hline
    \end{tabular}%
    }
    \end{table}

    \begin{table}[H]
        \resizebox{\textwidth}{!}{%
        \begin{tabular}{|l|l|l|l|l|l|}
        \hline
        \textbf{Caso de Uso} & \multicolumn{3}{l|}{Comprobar estado motor} & \multicolumn{2}{l|}{CU-10} \\ \hline
        \textbf{Actores} & \multicolumn{5}{l|}{ECU} \\ \hline
        \textbf{Tipo} & \multicolumn{5}{l|}{Primario, Esencial} \\ \hline
        \textbf{Referencias} & RF9 & \multicolumn{4}{l|}{-} \\ \hline
        \textbf{Precondición} & \multicolumn{5}{l|}{} \\ \hline
        \textbf{Postcondición} & \multicolumn{5}{l|}{Se habrá leído y comprobado el estado del motor} \\ \hline
        \textbf{Autor} & Sheila Martínez & \textbf{Fecha} & 23/06/2023 & \textbf{Versión} & v.1 \\ \hline
        \multicolumn{6}{|l|}{\cellcolor[HTML]{ECF4FF}Propósito} \\ \hline
        \multicolumn{6}{|l|}{La ECU obtiene el estado del motor} \\ \hline
        \multicolumn{6}{|l|}{\cellcolor[HTML]{ECF4FF}Resumen} \\ \hline
        \multicolumn{6}{|l|}{\begin{tabular}[c]{@{}l@{}}1. La ECU obtiene el estado del motor \end{tabular}} \\ \hline
        \end{tabular}%
        }
        \end{table}


\begin{table}[H]
    \resizebox{\textwidth}{!}{%
    \begin{tabular}{|l|l|l|l|l|l|}
    \hline
    \textbf{Caso de Uso} & \multicolumn{3}{l|}{Comprobar estado luces} & \multicolumn{2}{l|}{CU-11} \\ \hline
    \textbf{Actores} & \multicolumn{5}{l|}{ECU} \\ \hline
    \textbf{Tipo} & \multicolumn{5}{l|}{Primario, Esencial} \\ \hline
    \textbf{Referencias} & RF9 & \multicolumn{4}{l|}{-} \\ \hline
    \textbf{Precondición} & \multicolumn{5}{l|}{} \\ \hline
    \textbf{Postcondición} & \multicolumn{5}{l|}{Se habrá leído y comprobado el estado de las luces} \\ \hline
    \textbf{Autor} & Sheila Martínez & \textbf{Fecha} & 23/06/2023 & \textbf{Versión} & v.1 \\ \hline
    \multicolumn{6}{|l|}{\cellcolor[HTML]{ECF4FF}Propósito} \\ \hline
    \multicolumn{6}{|l|}{La ECU obtiene el estado de las luces} \\ \hline
    \multicolumn{6}{|l|}{\cellcolor[HTML]{ECF4FF}Resumen} \\ \hline
    \multicolumn{6}{|l|}{1. La ECU obtiene el estado de las luces} \\ \hline
    \end{tabular}%
    }
    \end{table}


    \begin{table}[H]
        \resizebox{\textwidth}{!}{%
        \begin{tabular}{|l|l|l|l|l|l|}
        \hline
        \textbf{Caso de Uso} & \multicolumn{3}{l|}{Comprobar temperatura batería} & \multicolumn{2}{l|}{CU-12} \\ \hline
        \textbf{Actores} & \multicolumn{5}{l|}{ECU} \\ \hline
        \textbf{Tipo} & \multicolumn{5}{l|}{Primario, Esencial} \\ \hline
        \textbf{Referencias} & RF9 & \multicolumn{4}{l|}{CU-17} \\ \hline
        \textbf{Precondición} & \multicolumn{5}{l|}{} \\ \hline
        \textbf{Postcondición} & \multicolumn{5}{l|}{Se habrá leído y comprobado el valor de la temperatura de la batería} \\ \hline
        \textbf{Autor} & Sheila Martínez & \textbf{Fecha} & 23/06/2023 & \textbf{Versión} & v.1 \\ \hline
        \multicolumn{6}{|l|}{\cellcolor[HTML]{ECF4FF}Propósito} \\ \hline
        \multicolumn{6}{|l|}{La ECU obtiene el valor de la temperatura de la batería y comprueba si es correcta.} \\ \hline
        \multicolumn{6}{|l|}{\cellcolor[HTML]{ECF4FF}Resumen} \\ \hline
        \multicolumn{6}{|l|}{\begin{tabular}[c]{@{}l@{}}1. La ECU obtiene el valor de la temperatura de la batería.\\ 2. Comprueba si el valor es correcto.\\ \\ 2.1 Si es incorrecto, enciende un piloto que lo indica.\end{tabular}} \\ \hline
        \end{tabular}%
        }
        \end{table}


        \begin{table}[H]
            \resizebox{\textwidth}{!}{%
            \begin{tabular}{|l|l|l|l|l|l|}
            \hline
            \textbf{Caso de Uso} & \multicolumn{3}{l|}{Leer voltaje sistema} & \multicolumn{2}{l|}{CU-13} \\ \hline
            \textbf{Actores} & \multicolumn{5}{l|}{ECU} \\ \hline
            \textbf{Tipo} & \multicolumn{5}{l|}{Primario, Real} \\ \hline
            \textbf{Referencias} & RF9 & \multicolumn{4}{l|}{-} \\ \hline
            \textbf{Precondición} & \multicolumn{5}{l|}{} \\ \hline
            \textbf{Postcondición} & \multicolumn{5}{l|}{Se habrá leído y comprobado el voltaje} \\ \hline
            \textbf{Autor} & Sheila Martínez & \textbf{Fecha} & 23/06/2023 & \textbf{Versión} & v.1 \\ \hline
            \multicolumn{6}{|l|}{\cellcolor[HTML]{ECF4FF}Propósito} \\ \hline
            \multicolumn{6}{|l|}{La ECU obtiene el voltaje del sistema} \\ \hline
            \multicolumn{6}{|l|}{\cellcolor[HTML]{ECF4FF}Resumen} \\ \hline
            \multicolumn{6}{|l|}{1. La ECU obtiene el voltaje del sistema} \\ \hline
            \end{tabular}%
            }
            \end{table}


\begin{table}[H]
    \resizebox{\textwidth}{!}{%
    \begin{tabular}{|l|l|l|l|l|l|}
    \hline
    \textbf{Caso de Uso} & \multicolumn{3}{l|}{Calcular consumo} & \multicolumn{2}{l|}{CU-14} \\ \hline
    \textbf{Actores} & \multicolumn{5}{l|}{ECU} \\ \hline
    \textbf{Tipo} & \multicolumn{5}{l|}{Primario, Esencial} \\ \hline
    \textbf{Referencias} & RF9 & \multicolumn{4}{l|}{CU-13} \\ \hline
    \textbf{Precondición} & \multicolumn{5}{l|}{Se debe haber leído el voltaje de la batería} \\ \hline
    \textbf{Postcondición} & \multicolumn{5}{l|}{Se habrá calculado el consumo} \\ \hline
    \textbf{Autor} & Sheila Martínez & \textbf{Fecha} & 23/06/2023 & \textbf{Versión} & v.1 \\ \hline
    \multicolumn{6}{|l|}{\cellcolor[HTML]{ECF4FF}Propósito} \\ \hline
    \multicolumn{6}{|l|}{La ECU obtiene el consumo instantáneo} \\ \hline
    \multicolumn{6}{|l|}{\cellcolor[HTML]{ECF4FF}Resumen} \\ \hline
    \multicolumn{6}{|l|}{1. La ECU obtiene el consumo instantáneo} \\ \hline
    \end{tabular}%
    }
    \end{table}

\begin{table}[H]
    \resizebox{\textwidth}{!}{%
    \begin{tabular}{|l|l|l|l|l|l|}
    \hline
    \textbf{Caso de Uso} & \multicolumn{3}{l|}{Calcular autonomía} & \multicolumn{2}{l|}{CU-15} \\ \hline
    \textbf{Actores} & \multicolumn{5}{l|}{ECU} \\ \hline
    \textbf{Tipo} & \multicolumn{5}{l|}{Primario, Esencial} \\ \hline
    \textbf{Referencias} & RF9 & \multicolumn{4}{l|}{CU-14} \\ \hline
    \textbf{Precondición} & \multicolumn{5}{l|}{Se debe saber el voltaje máximo de la batería y el consumo} \\ \hline
    \textbf{Postcondición} & \multicolumn{5}{l|}{Se habrá calculado la autonomía} \\ \hline
    \textbf{Autor} & Sheila Martínez & \textbf{Fecha} & 23/06/2023 & \textbf{Versión} & v.1 \\ \hline
    \multicolumn{6}{|l|}{\cellcolor[HTML]{ECF4FF}Propósito} \\ \hline
    \multicolumn{6}{|l|}{La ECU obtiene la autonomía del sistema} \\ \hline
    \multicolumn{6}{|l|}{\cellcolor[HTML]{ECF4FF}Resumen} \\ \hline
    \multicolumn{6}{|l|}{La ECU obtiene la autonomía del sistema en tiempo y porcentaje} \\ \hline
    \end{tabular}%
    }
    \end{table}


\begin{table}[H]
    \resizebox{\textwidth}{!}{%
    \begin{tabular}{|l|l|l|l|l|l|}
    \hline
    \textbf{Caso de Uso} & \multicolumn{3}{l|}{Calcular carga batería} & \multicolumn{2}{l|}{CU-16} \\ \hline
    \textbf{Actores} & \multicolumn{5}{l|}{ECU} \\ \hline
    \textbf{Tipo} & \multicolumn{5}{l|}{Primario, Esencial} \\ \hline
    \textbf{Referencias} & RF9 & \multicolumn{4}{l|}{CU-13} \\ \hline
    \textbf{Precondición} & \multicolumn{5}{l|}{Se debe saber el voltaje máximo de la batería y el voltaje actual} \\ \hline
    \textbf{Postcondición} & \multicolumn{5}{l|}{Se habrá calculado la carga de la batería} \\ \hline
    \textbf{Autor} & Sheila Martínez & \textbf{Fecha} & 23/06/2023 & \textbf{Versión} & v.1 \\ \hline
    \multicolumn{6}{|l|}{\cellcolor[HTML]{ECF4FF}Propósito} \\ \hline
    \multicolumn{6}{|l|}{La ECU obtiene la carga de la batería} \\ \hline
    \multicolumn{6}{|l|}{\cellcolor[HTML]{ECF4FF}Resumen} \\ \hline
    \multicolumn{6}{|l|}{La ECU obtiene la carga de la batería en porcentaje} \\ \hline
    \end{tabular}%
    }
    \end{table}

%    \begin{table}[]
%        \resizebox{\textwidth}{!}{%
%        \begin{tabular}{|l|l|l|l|l|l|}
%        \hline
%        \textbf{Caso de Uso} & \multicolumn{3}{l|}{Variar voltaje} & \multicolumn{2}{l|}{CU-16} \\ \hline
%        \textbf{Actores} & \multicolumn{5}{l|}{ECU} \\ \hline
%        \textbf{Tipo} & \multicolumn{5}{l|}{Primario, Esencial} \\ \hline
%        \textbf{Referencias} & Requisitos que se pueden incluir dentro de este cu & \multicolumn{4}{l|}{Cu que tiene relacion} \\ \hline
%        \textbf{Precondición} & \multicolumn{5}{l|}{Se debe tener un valor al que variar el voltaje} \\ \hline
%        \textbf{Postcondición} & \multicolumn{5}{l|}{Se habrá variado el voltaje} \\ \hline
%        \textbf{Autor} & Sheila Martínez & \textbf{Fecha} & 23/06/2023 & \textbf{Versión} & v.1 \\ \hline
%        \multicolumn{6}{|l|}{\cellcolor[HTML]{ECF4FF}Propósito} \\ \hline
%        \multicolumn{6}{|l|}{La ECU varía el voltaje en relación al valor recibido} \\ \hline
%        \multicolumn{6}{|l|}{\cellcolor[HTML]{ECF4FF}Resumen} \\ \hline
%        \multicolumn{6}{|l|}{La ECU varía el voltaje del sistema conforme al valor que se le ha dado, tras comprobar que este valor es correcto.} \\ \hline
%        \end{tabular}%
%        }
%       \end{table}


\begin{table}[H]
    \resizebox{\textwidth}{!}{%
    \begin{tabular}{|l|l|l|l|l|l|}
    \hline
    \textbf{Caso de Uso} & \multicolumn{3}{l|}{Mostrar datos} & \multicolumn{2}{l|}{CU-17} \\ \hline
    \textbf{Actores} & \multicolumn{5}{l|}{ECU} \\ \hline
    \textbf{Tipo} & \multicolumn{5}{l|}{Primario, Esencial} \\ \hline
    \textbf{Referencias} & RF9 & \multicolumn{4}{l|}{-} \\ \hline
    \textbf{Precondición} & \multicolumn{5}{l|}{Se deben tener calculados y leídos todos los datos del sistema} \\ \hline
    \textbf{Postcondición} & \multicolumn{5}{l|}{Se mostrarán los datos del sistema} \\ \hline
    \textbf{Autor} & Sheila Martínez & \textbf{Fecha} & 23/06/2023 & \textbf{Versión} & v.1 \\ \hline
    \multicolumn{6}{|l|}{\cellcolor[HTML]{ECF4FF}Propósito} \\ \hline
    \multicolumn{6}{|l|}{La ECU muestra los datos del sistema} \\ \hline
    \multicolumn{6}{|l|}{\cellcolor[HTML]{ECF4FF}Resumen} \\ \hline
    \multicolumn{6}{|l|}{\begin{tabular}[c]{@{}l@{}}La ECU muestra por pantalla todos los datos obtenidos por los sensores\\  para que el conductor pueda servirse de ellos\end{tabular}} \\ \hline
    \end{tabular}%
    }
\end{table}

\newpage
\section{Diagramas de casos de uso}

Una vez realizados los casos de uso, los \textbf{diagramas de uso} nos permiten tener una visión general del conjunto de subsistemas, así como su interacción con los actores: 


\begin{figure}[H]
    \centering
    \includegraphics[width=1.1\textwidth]{imagenes/diagrama_CU_1.png}
\end{figure}

\begin{figure}[H]
    \centering
    \includegraphics[width=1\textwidth]{imagenes/diagrama_CU_2.png}
\end{figure}

\begin{figure}[H]
    \centering
    \includegraphics[width=1\textwidth]{imagenes/diagrama_CU_3.png}
\end{figure}


\section{Requisitos}

Los requisitos nos van a permitir comprender y designar cuáles serán las capacidades y funciones de nuestro producto para resolver una necesidad expresada por el usuario. Existen dos tipos principales, que se exponen a continuación.

\subsection*{Requisitos funcionales}

Los requisitos funcionales son aquellos requisitos que describen las funciones que debe realizar el sistema, es decir, la interacción entre este y su entorno, e indican cuál debe ser la reacción al sistema ante una determinada entrada. 

\begin{table}[H]
    \resizebox{\textwidth}{!}{%
    \begin{tabular}{|l|l|}
    \hline
    \textbf{Nº de RF} & 1 \\ \hline
    \textbf{Nombre} & Encender el motor \\ \hline
    \textbf{Descripción} & Como conductor, quiero poder encender el motor del vehículo \\ \hline
    \textbf{Prioridad} & Alta \\ \hline
    \textbf{Entrada} & Una señal \\ \hline
    \textbf{Prerrequisitos} & El motor debe estar apagado \\ \hline
    \textbf{Procesamiento} & El usuario pulsa el botón de arranque y el motor se activa \\ \hline
    \rowcolor[HTML]{FFFFFF} 
    \textbf{Postcondición} & - \\ \hline
    \end{tabular}%
    }
    \end{table}

\begin{table}[H]
    \resizebox{\textwidth}{!}{%
    \begin{tabular}{|l|l|}
    \hline
    \textbf{Nº de RF} & 2 \\ \hline
    \textbf{Nombre} & Apagar el motor \\ \hline
    \textbf{Descripción} & Como conductor, quiero poder apagar el motor del vehículo \\ \hline
    \textbf{Prioridad} & Alta \\ \hline
    \textbf{Entrada} & Una señal \\ \hline
    \textbf{Prerrequisitos} & El motor debe estar encendido \\ \hline
    \textbf{Procesamiento} & El usuario pulsa el botón de arranque y el motor se apaga \\ \hline
    \rowcolor[HTML]{FFFFFF} 
    \textbf{Postcondición} & - \\ \hline
    \end{tabular}%
    }
    \end{table}


\begin{table}[H]
    \resizebox{\textwidth}{!}{%
    \begin{tabular}{|l|l|}
    \hline
    \textbf{Nº de RF} & 3 \\ \hline
    \textbf{Nombre} & Encender las luces \\ \hline
    \textbf{Descripción} & Como conductor, quiero poder encender las luces del vehículo \\ \hline
    \textbf{Prioridad} & Media \\ \hline
    \textbf{Entrada} & Una señal \\ \hline
    \textbf{Prerrequisitos} & Las luces deben estar apagadas \\ \hline
    \textbf{Procesamiento} & El usuario pulsa el botón de encendido de luces y las luces se encienden \\ \hline
    \rowcolor[HTML]{FFFFFF} 
    \textbf{Postcondición} & - \\ \hline
    \end{tabular}%
    }
    \end{table}


\begin{table}[H]
    \resizebox{\textwidth}{!}{%
    \begin{tabular}{|l|l|}
    \hline
    \textbf{Nº de RF} & 4 \\ \hline
    \textbf{Nombre} & Apagar las luces \\ \hline
    \textbf{Descripción} & Como conductor, quiero poder apagar las luces del vehículo \\ \hline
    \textbf{Prioridad} & Media \\ \hline
    \textbf{Entrada} & Una señal \\ \hline
    \textbf{Prerrequisitos} & Las luces deben estar encendidas \\ \hline
    \textbf{Procesamiento} & El usuario pulsa el botón de apagado de luces y las luces se apagan \\ \hline
    \rowcolor[HTML]{FFFFFF} 
    \textbf{Postcondición} & - \\ \hline
    \end{tabular}%
    }
    \end{table}

\begin{table}[H]
    \resizebox{\textwidth}{!}{%
    \begin{tabular}{|l|l|}
    \hline
    \textbf{Nº de RF} & 5 \\ \hline
    \textbf{Nombre} & Encender intermitentes \\ \hline
    \textbf{Descripción} & Como conductor, quiero poder encender el intermitente de un lado del vehículo \\ \hline
    \textbf{Prioridad} & Media \\ \hline
    \textbf{Entrada} & Una señal \\ \hline
    \textbf{Prerrequisitos} & El intermitente debe estar en su posición inicial (apagado) \\ \hline
    \textbf{Procesamiento} & \begin{tabular}[c]{@{}l@{}}El usuario mueve la palanca hacia el lado determinado y el intermitente de ese\\ lado se enciende\end{tabular} \\ \hline
    \rowcolor[HTML]{FFFFFF} 
    \textbf{Postcondición} & - \\ \hline
    \end{tabular}%
    }
    \end{table}

\begin{table}[H]
    \resizebox{\textwidth}{!}{%
    \begin{tabular}{|l|l|}
    \hline
    \textbf{Nº de RF} & 6 \\ \hline
    \textbf{Nombre} & Apagar intermitentes \\ \hline
    \textbf{Descripción} & Como conductor, quiero poder apagar el intermitente de un lado del vehículo \\ \hline
    \textbf{Prioridad} & Media \\ \hline
    \textbf{Entrada} & Una señal \\ \hline
    \textbf{Prerrequisitos} & El intermitente no debe estar en su posición inicial (debe estar encendido) \\ \hline
    \textbf{Procesamiento} & El usuario mueve la palanca hacia su posición inicial y los intermitentes se apagan \\ \hline
    \rowcolor[HTML]{FFFFFF} 
    \textbf{Postcondición} & - \\ \hline
    \end{tabular}%
    }
    \end{table}

\begin{table}[H]
    \resizebox{\textwidth}{!}{%
    \begin{tabular}{|l|l|}
    \hline
    \textbf{Nº de RF} & 7 \\ \hline
    \textbf{Nombre} & Cambiar estado de la luz de freno \\ \hline
    \textbf{Descripción} & Como conductor, quiero poder pisar el freno y que se encienda la luz correspondiente \\ \hline
    \textbf{Prioridad} & Alta \\ \hline
    \textbf{Entrada} & Una señal \\ \hline
    \textbf{Prerrequisitos} & - \\ \hline
    \textbf{Procesamiento} & \begin{tabular}[c]{@{}l@{}}La luz de freno se enciende cuando el conductor pisa el freno, y se apaga cuando\\ deja de pulsarlo\end{tabular} \\ \hline
    \rowcolor[HTML]{FFFFFF} 
    \textbf{Postcondición} & - \\ \hline
    \end{tabular}%
    }
    \end{table}

\begin{table}[H]
    \resizebox{\textwidth}{!}{%
    \begin{tabular}{|l|l|}
    \hline
    \textbf{Nº de RF} & 8 \\ \hline
    \textbf{Nombre} & Cambiar velocidad \\ \hline
    \textbf{Descripción} & Como conductor, quiero poder pisar el acelerador  y que el motor acelere \\ \hline
    \textbf{Prioridad} & Alta \\ \hline
    \textbf{Entrada} & Una señal \\ \hline
    \textbf{Prerrequisitos} & - \\ \hline
    \textbf{Procesamiento} & \begin{tabular}[c]{@{}l@{}}El motor aumenta su velocidad proporcionalmente al recorrido que haya \\ hecho el pedal de acelerador al ser pisado por el conductor\end{tabular} \\ \hline
    \rowcolor[HTML]{FFFFFF} 
    \textbf{Postcondición} & - \\ \hline
    \end{tabular}%
    }
    \end{table}

\begin{table}[H]
    \resizebox{\textwidth}{!}{%
    \begin{tabular}{|l|l|}
    \hline
    \textbf{Nº de RF} & 9 \\ \hline
    \textbf{Nombre} & Comprobar datos del sistema \\ \hline
    \textbf{Descripción} & \begin{tabular}[c]{@{}l@{}}La ECU debe poder leer los datos de los diferentes subsistemas en \\ todo momento y detectar cuándo algún valor es incorrecto.\end{tabular} \\ \hline
    \textbf{Prioridad} & Alta \\ \hline
    \textbf{Entrada} & Un conjunto de datos recibidos de los sensores \\ \hline
    \textbf{Prerrequisitos} & - \\ \hline
    \textbf{Procesamiento} & \begin{tabular}[c]{@{}l@{}}La ECU monitoriza en intervalos determinados los valores del resto \\ de subsistemas y comprueba si algún valor es incorrecto, tras lo cual \\ muestra un piloto o señal luminosa al usuario que lo indica.\end{tabular} \\ \hline
    \rowcolor[HTML]{FFFFFF} 
    \textbf{Postcondición} & - \\ \hline
    \end{tabular}%
    }
    \end{table}

\begin{table}[H]
    \resizebox{\textwidth}{!}{%
    \begin{tabular}{|l|l|}
    \hline
    \textbf{Nº de RF} & 10 \\ \hline
    \textbf{Nombre} & Mostrar datos del sistema \\ \hline
    \textbf{Descripción} & \begin{tabular}[c]{@{}l@{}}La ECU debe poder mostrar en cada momento los datos obtenidos de \\ los sensores, ya procesados.\end{tabular} \\ \hline
    \textbf{Prioridad} & Media \\ \hline
    \textbf{Entrada} & Un conjunto de datos recibidos de los sensores \\ \hline
    \textbf{Prerrequisitos} & - \\ \hline
    \textbf{Procesamiento} & \begin{tabular}[c]{@{}l@{}}La ECU muestra por pantalla los valores obtenidos, formateados para\\ que el conductor pueda comprenderlos de manera sencilla.\end{tabular} \\ \hline
    \rowcolor[HTML]{FFFFFF} 
    \textbf{Postcondición} & - \\ \hline
    \end{tabular}%
    }
    \end{table}
\newpage


\subsection*{Requisitos no funcionales}

Los requisitos no funcionales, o \textbf{atributos de calidad}, describen cualidades o restricciones del sistema, pero nunca funciones que este realiza. Estos requisitos son los que verifican la calidad del software y del proyecto en general.

\begin{itemize}
    \item \textbf{RNF-1 - Facilidad de uso:} El proyecto deberá estar documentado de manera clara y concisa, sin ambigüedades. El código deberá ser legible, y las variables deben ser representativas.
    \item \textbf{RNF-2 - Fiabilidad:} El sistema debe tener tolerancia a fallos, así como disponibilidad y capacidad de recuperación. Debe existir previsión ante los posibles errores que se pueden dar durante el funcionamiento del sistema, y una representación clara que indique el código de error.
    \item \textbf{RNF-3 - Rendimiento:} Al ser un sistema que, en un entorno real, puede poner en riesgo vidas humanas con un mal funcionamiento, los tiempos de respuesta deben estar acotados a lo mínimo posible. El uso de los recursos debe ser eficiente, y la prioridad de las funciones con mayor importancia debe respetarse en cualquier situación.
    \item \textbf{RNF-4 - Capacidad de soporte:} Se evitará en la medida de lo posible variar demasiado las funciones planteadas pero, en el caso de que tenga que haber algún cambio, estos estarán representados con claridad en la nueva versión. El proyecto tener una organización modular para que las nuevas funcionalidades (si las hubiera) se puedan adaptar con facilidad en el sistema base. 
    \item \textbf{RNF-5 - Implementación}: [pend]
    \item \textbf{RNF-6 - Ámbito legal}: [oend]
\end{itemize}




	\input{secciones/05_analisis}

	\input{secciones/06_planificacion}


	\chapter{Implementación}

\fancyhead[R]{7. Implementación}

\noindent\fbox{
	\parbox{\textwidth}{
    En este capítulo se definen las acciones realizadas en cada etapa del proyecto, así como los problemas que hayan podido surgir, y la experiencia una vez finalizada cada fase. 
	}
}

\section{Desarrollo en Arduino}

Para desarrollar las funciones necesarias en Arduino, se trabajará con una protoboard para realizar el prototipo y poder comprobar que todas las funcionalidades trabajan de la forma correcta. La versión de Arduino IDE en la que se realizará este proyecto es la \textbf{v2.1}, lanzada el 24 de abril de 2023. 

\subsection{Desarrollo de funciones secuenciales sin FREERTOS}

En el primer apartado del desarrollo en arduino, se programarán funciones simples para cada uno de los subsistemas, sin entrada del usuario ni de programas externos, para facilitar la comprensión de dichas funciones y agilizar la temporización de las funciones con FreeRTOS. 

\subsubsection{Sección de luces}

Para realizar el encendido y apagado de luces, lo primero es declarar los pines digitales necesarios como OUTPUT, para después variar su estado con \textbf{digitalWrite()}. El retardo, si fuese necesario, se implementará con un tiempo de 200ms entre estados mediante la función \textbf{delay}.

A continuación se muestra el circuito utilizado para comprobar que todas las funciones que se implementarán a continuación funcionan correctamente: 

\begin{figure}[H]
    \centering
    \includegraphics[width=1\textwidth]{imagenes/diagramas/luces_ard.png}
    \caption{Circuito de las luces. Elaboración propia}
\end{figure}

A partir de ahora, se distinguirá en dos tipos de funciones:

\paragraph*{Funciones de luces con parpadeo}
Estas funciones son aquellas en las que las luces varían entre encendidas y apagadas con un retardo acotado, siendo estas las necesarias para el funcionamiento y las luces de emergencia.

\begin{figure}[H]
    \centering
    \includegraphics[width=0.9\textwidth]{imagenes/diagramas/turn_base.png}
    \caption{Diagrama de flujo de la función de luces intermitentes. Elaboración propia}
\end{figure}


\paragraph*{Funciones de luces fijas}
Las luces en estas funciones se mantienen fijas en el estado que se requiera, ya sea apagadas o encendidas, al recibir entrada del usuario (implementado directamente en la próxima sección mediante el serial). Entran en este apartado las luces de freno y las luces neutrales o cortas. 


\begin{figure}[H]
    \centering
    \includegraphics[width=0.9\textwidth]{imagenes/diagramas/brake_base.png}
    \caption{Diagrama de flujo de la función de luces de freno. Elaboración propia}
\end{figure}







\subsubsection{Sección de temperatura}

Para el subsistema que controla la temperatura, debemos tener en cuenta los dos termistores que se van a utilizar y cómo obtener el valor real de temperatura una vez tenemos el valor del sensor. Para ello, se debe estudiar el \textit{datasheet} del componente. A continuación se expone el circuito a utilizar:

\begin{figure}[h]
    \centering
    \includegraphics[width=0.7\textwidth]{imagenes/diagramas/temp_ard.png}
    \caption{Circuito de la temperatura. Elaboración propia}
\end{figure}



Para calcular la temperatura real, se deben calcular los valores necesarios mediante la ecuación de /textbf{Steinhart-Hart}, no sin antes obtener las variables \textbf{A, B } \textbf{C} requeridas para dicha ecuación \cite{termistor}.

\begin{align*}
    \frac{1}{T} = A + B\ln{R} + C(\ln{R})^{3}
\end{align*}

Donde las variables tienen los siguientes valores:
\begin{itemize}
    \item \textbf{A}: 1.11492089e-3
    \item \textbf{B}: 2.372075385e-4
    \item \textbf{C}: 6.954079529e-8
    \item \textbf{R}: R\_externa*V / V\_in-V
\end{itemize}

Una vez se tienen todos los valores necesarios, únicamente resta incluir todos estos cálculos en una función para la lectura, y crear otra función que se encargue de escribir el valor de temperatura real en el puerto serial para su visualización.

\begin{figure}[H]
    \centering
    \includegraphics[width=1\textwidth]{imagenes/diagramas/temp_base.png}
    \caption{Diagrama de flujo de la función de temperatura. Elaboración propia}
\end{figure}




\subsubsection{Sección de motor}

Para el control de los motores se utilizará la placa \textbf{L928N}, como se expuso anteriormente. Esta placa simplifica el circuito y permite que no existan conexiones directas desde el Arduino Mega a los motores, pues puede provocar cortocircuitos. 

\begin{figure}[H]
    \centering
    \includegraphics[width=0.6\textwidth]{imagenes/diagramas/motor_ard.png}
    \caption{Circuito de los motores. Elaboración propia}
\end{figure}

En esta función base no se utilizarán los pines de comunicaciones, por lo que mantendremos los jumper. El objetivo es únicamente comprobar que se puede encender y apagar los motores, no variar su velocidad (función que se implementará en FreeRTOS directamente, como indica el esquema inferior).

Para ello, se deben definir los pines digitales con señal PWM de ambos motores como OUTPUT, y escribir en ellos HIGH o LOW para encender o apagarlos. El control de dirección es sencillo, pues se ha realizado conectando los cables de positivo y negativo de manera inversa entre ambos motores. 

\begin{figure}[H]
    \centering
    \includegraphics[width=1\textwidth]{imagenes/diagramas/motor_base.png}
    \caption{Diagrama de flujo de la función de velocidad del motor. Elaboración propia}
\end{figure}



\subsubsection{Función setup}

En esta función, que realiza la misma tarea que en Arduino secuencial, se va a declarar todas las variables globales, manejadores, colas, semáforos y tareas necesarias para el funcionamiento del programa. 

Además de estos pasos, también se deberá crear las instancias de tareas de lectura de puerto serie y temperaturas, que inician su ejecución al comenzar la ejecución del programa. 

La estructura de creación de una tarea en FreeRTOS es la siguiente. 

\begin{verbatim}
    xTaskCreate(
        NombreTarea,
        "NombreParaProgramador",
        TamanioPila,
        Argumentos,
        Prioridad,
        &manejador);
\end{verbatim}

Si bien todos los argumentos de la función \textbf{xTaskCreate} son importantes, los manejadores o \textit{handler} son decisivos para cualquier proyecto, pues permiten gestionar el estado de la función que controlan desde cualquier otro punto del código, ya sea otra función u otra tarea. 

\subsection{Desarrollo de funciones concurrentes con FREERTOS}

En este apartado se desarrollará la versión concurrente de las funciones que se programaron anteriormente, añadiendo la comunicación con el puerto serie y gestionando el uso de recursos compartidos entre los distintos procesos. 


\subsubsection{Función de manejo del serial}

Esta función será la piedra angular del programa, pues será la encargada de recibir entrada del usuario, procesar su petición y, si fuera necesario, mostrar los datos en el puerto serie (o en la interfaz gráfica en los siguientes apartados).

Se ha inicializado un semáforo que se encargue de asegurar que todas las lecturas y escrituras sobre el puerto serie estén protegidas, evitando así la pérdida de datos o pulsaciones del usuario. A continuación se detalla la sintaxis del bloqueo que impone sobre el semáforo la función del serial para facilitar la comprensión.

\begin{verbatim}
    if (xSemaphoreTake(serial_sem, (TickType_t)5) == pdTRUE) {
        value_serial = Serial.read();
        xSemaphoreGive(serial_sem);
        vTaskDelay(20 / portTICK_PERIOD_MS);
      }  
\end{verbatim}

Como se puede observar en el código, se realizan los siguientes pasos:

\begin{enumerate}
    \item La función comprueba que el semáforo no esté bloqueado en otro punto del código y, si no lo está, se adueña de él.
    \item Se realiza la lectura del serial pertinente.
    \item Se devuelve el semáforo y se añade un retardo para asegurar que no se produzca una condición de carrera. 
\end{enumerate}

Una vez se ha obtenido el valor del serial introducido por el usuario y/o la interfaz, se gestionará mediante un switch para evitar la sobrecarga del puerto serie al intentar leer de él todas las funciones. Una vez detectado el valor que corresponde a una de estas funciones, se realizarán las acciones asociadas a dicha señal, que normalmente son las siguientes:

\begin{enumerate}
    \item Comprobación del estado de la función, si el sistema a manejar está encendido o apagado.
    \item Comprobación con dependencias de otras funciones que puedan ser incompatibles con la que se llama mediante el serial.
    \item Creación o destrucción de la tarea asociada teniendo en cuenta el estado de la misma. 
    \item Limpieza de pines asociados a la tarea si fuese necesario. 
\end{enumerate}



\subsubsection{Funciones de luces}

Las funciones asociadas a la iluminación no han variado demasiado respecto a las secuenciales, aunque se han añadido restricciones entre luces de emergencia e intermitentes para que no se desincronice el parpadeo de los LEDs. También se ha acotado el retardo respecto al tiempo de acceso a los recursos mediante la variable \textbf{portTICK\_PERIOD\_MS}.

\subsubsection{Funciones de temperatura}

Las funciones y tareas referentes a las temperaturas de motor y batería han sufrido grandes cambios, entre ellos la modularización de las instrucciones.

Para realizar la lectura desde el sensor se ha creado una función que se encargue de leer el valor del sensor y calcular la temperatura real. Dicha función es llamada por la tarea de lectura de temperatura, que está ejecutándose desde que se carga el programa en la placa. 

La comunicación entre las tareas de lectura y escritura ahora se realizará mediante una cola para minimizar el uso de variables globales y agilizar el tiempo de respuesta. 

Se ha añadido también una codificación de la temperatura para distinguir entre los valores de los dos termistores a la hora de desarrollar la interfaz y distribuir los datos recibidos de manera correcta.

\subsubsection{Funcion del motor}

La función de control de velocidad del motor ha sido implementada mediante una tarea, que recibe las pulsaciones '+' y '-' para aumentar y disminuir el voltaje en doce tramos. Se ha añadido un valor inicial para designar el voltaje mínimo que requieren los motores para funcionar de manera correcta. 

El control del máximo y mínimo de velocidad se harán mediante interfaz para simplificar la comunicación, por lo que esta versión todavía no posee mecanismos de limitación de velocidad \textit{per se}, aunque al estar trabajando en un puerto analógico el máximo valor será siempre 255, por lo que existe esa restricción que evita que pueda suministrarse un exceso de voltaje al sistema. 

\begin{figure}[H]
    \centering
    \includegraphics[width=1\textwidth]{imagenes/diagramas/Diagrama_FREERTOS.png}
    \caption{Diagrama de flujo del código con comunicación. Elaboración propia}
\end{figure}


\section{Desarrollo en Processing}

Para diseñar la interfaz en Processing se utilizará la librería ControlP5, ya mencionada anteriormente, por su facilidad de implementación de botones, ruedas y \textit{sliders}. Sería posible diseñar estas funcionalidades desde cero, ya que Processing es muy flexible, pero el uso de ControlP5 simplifica el desarrollo enormemente. 

\subsection{Desarrollo de interfaz de usuario base sin comunicaciones serie}


Antes de comenzar a programarla, lo mejor es realizar un esquema simple indicando dónde debe ir cada función y cada sección, por lo que se realiza un boceto en GIMP con dicha información:

\begin{figure}[H]
    \centering
    \includegraphics[width=1\textwidth]{imagenes/cargui_base.png}
    \caption{Boceto de la interfaz gráfica. Elaboración propia}
\end{figure}

Una vez se tiene una idea general de lo que se va a realizar, es el momento de programar y diseñar el boceto. Se segmentará el desarrollo en secciones para menor granularidad de tareas.

\subsubsection{Panel de luces}

El panel de luces es sencillo, pues únicamente constará de cinco botones. La programación de dichos botones se realizará mediante la función \textbf{addButton}. A continuación se expone el código para la declaración de uno de estos botones:

\begin{verbatim}
    cp5.addButton("IntermitenteIZQ")
    .setSize(180,80)
    .setPosition(20, 60)
    .setFont(font)
    .setLabel("Intermitente \n    izquierdo")
\end{verbatim}

Si bien todas las propiedades son importantes, el nombre que se le da al botón, en este caso "intermitenteIZQ", es extremadamente importante, pues será el nombre de la función que se llamará cuando se pulse el botón, y permita asociarle una acción. Dicha función se implementará durante la próxima sección.

\subsubsection{Panel general}

El objetivo de este panel será proveer de información al usuario con un solo vistazo mediante iconos que se iluminarán dependiendo de las circunstancias, véase un intermitente encendido, exceso de temperatura, etcétera. También incluirá datos relevantes de la versión de la interfaz como su autoría y la versión en la que se encuentra. 

Durante el desarrollo de este panel, surgió un problema a la hora de visualizar los iconos encendidos y apagados, pues no se podía eliminar una imagen y mostrar la del estado en el que se encontraba el sistema. Para solucionar esto se añadió una jerarquía de imágenes, mostrando por defecto el icono de apagado, y superponiendo el icono de encendido cuando fuera necesario.

\subsubsection{Panel de temperatura}

Este panel es posiblemente el más complejo de implementar. Si bien existen otras herramientas y librerías para la visualización de valores de temperatura, se va a adaptar una función de ControlP5 para mantener el estilo coherente con el resto de funcionalidades. 

Se trabajó con el objeto \textbf{knob}, que normalmente se utiliza para seleccionar un valor en un rango mediante el deslizamiento del ratón. Dicho valor, que normalmente es variable por entrada del usuario, se tomará de los datos que envíe la placa por el puerto serie en próximas versiones, además de eliminar la posibilidad de introducir el valor de manera manual. 

En este momento surgieron varios problemas con el tratamiento de temperaturas, pues ambos visores mostraban valores iguales. Esto supuso un cambio en la función de temperatura, pues el problema era que no existía diferenciación entre los valores que mandaban ambos termistores a la GUI. Para poder subsanar este error se introdujo un caracter al final del valor de temperatura de cada uno de los termistores, pudiendo así identificar a cuál de ellos correspondía el valor recibido. También se añadió un método para poder eliminar ese caracter distintivo y recuperar el tipo flotante de la temperatura, y así poder mostrarlo en el visor. 

Una vez implementados los dos visores de temperatura, se añadieron botones para encender y apagar la escritura de temperaturas, además de añadir la visualización numérica de los valores.

\subsubsection{Panel de motor}

Para realizar la sección de control de velocidad de motor se implementará un visor desde cero, realizado mediante un array de figuras (en este caso rectángulos) que se iluminarán dependiendo del valor de velocidad que tenga el sistema en el momento. Se podrá aumentar o disminuir la velocidad pulsando los botones '+' o '-', así como encender y apagar el motor de la misma manera. 

El voltaje del motor se mostrará de manera aproximada debido a la imposibilidad de incluir un sensor de voltaje en el sistema, basándose en el mínimo y máximo voltaje con el que trabajan los motores que se utilizan, y el voltaje que aporta la batería. 

\begin{figure}[H]
    \centering
    \includegraphics[width=1\textwidth]{imagenes/cargui_simple.png}
    \caption{Esqueleto de la interfaz sin funcionalidad. Elaboración propia}
\end{figure}



\subsection{Desarrollo interfaz de usuario con comunicaciones serie}

El último paso de la implementación del código se hará trabajando mayormente en la interfaz. Se añadirá lectura del puerto serie, se mejorarán los grafismos y se incluirán las funciones que no se han podido añadir antes, dotando de funcionalidad a todo el sistema. 

\subsubsection{Comunicación con el puerto serie}

El primer paso para comunicar Arduino y Processing es la conexión de este último al puerto serie, utilizando la librería \textbf{processing.serial}. Para poder enviar los avisos de pulsaciones de botones se trabajará con las funciones asociadas a estos, enviando el valor correspondiente al botón hacia la placa, donde será procesado.

Por otro lado, los datos recibidos de la placa serán procesados y mostrados en sus respectivos lugares gracias a la codificación que se le ha incluido, por ejemplo, a las temperaturas. 

\subsubsection{Mejora visual de la interfaz}

Para evitar que la interfaz quede demasiado escueta se han diseñado imágenes de cada uno de los botones, así como iconos para mostrar el estado de los sistemas en el panel general. También se incluye un grafismo de un vehículo en el que se mostrarán las distintas luces encendidas en tiempo real, emulando lo que se vería en una maqueta real. 

Además, los visores de temperatura ahora muestran mediante un degradado el rango de temperatura en el que se encuentran.

A continuación se muestra una imagen del programa con plena funcionalidad:

\begin{figure}[H]
    \centering
    \includegraphics[width=1\textwidth]{imagenes/cargui_final.png}
    \caption{Interfaz finalizada con conexión a arduino. Elaboración propia}
\end{figure}


\section{Depuración, limpieza y documentación del código fuente}

Una vez realizado todo el código para la interfaz y la placa, se debe realizar una depuración del mismo. Para comprobar el funcionamiento correcto de la aplicación se han realizado pruebas de carga, iniciando todas las funciones en la intefaz y comprobando que no existan fallos en la concurrencia. Al realizar dichas pruebas han aparecido problemas con las temperaturas y las luces que, al ejecutarse simultáneamente, provocaban que la temperatura variase de uno a dos grados al encenderse, por ejemplo, los intermitentes. Dicho problema existía por el hecho de estar conectadas a la misma tierra, lo cual generaba una diferencia de potencial que cambiaba el valor de la temperatura. Este error se ha visto subsanado al añadir otra tierra diferente para ambos termistores, independientemente de las luces. 

También han surgido problemas por la relación entre las funciones de intemitentes y luces de emergencia al trabajar ambas sobre los mismos pines. El problema se presentaba cuando las luces de emegencia estaban encendidas y se activaba uno de los dos intermitentes, pues se desincronizaban las luces de ambas direcciones. La manera de arreglar este problema ha sido añadir un control en la función de serial indicando el estado de las luces de emergencia y, si estas están activas, no crear la función de los intemitentes. 

Tras solucionar los errores y realizar la depuración del código, se ha organizado en secciones para simplifican la lectura del mismo, así como uniformar el formato de las funciónes y los nombres de las variables. También se ha añadido un archivo de cabeceras para todas las variables y declaración de funciones, semáforos, colas y tareas, además de documentar dicho archivo para indicar las variables que reciben y retornan las funciones y tareas, así como una breve descripción de su funcionamiento y/o utilidad. 


\section{Montaje en maqueta} 

En este apartado se detallarán los pasos seguidos para integrar todos los componentes en una maqueta, así como todos los problemas y errores que se han cometido durante el montaje. La comunicación con la maqueta se realizará por cable, captando las órdenes desde el ordenador mediante los botones de la interfaz, y siendo esta quien enviará la orden al programa por el puerto serie, configurado a 9600 baudios, para ser recibida por medio de un USB por la placa, que ejecutará las acciones pertinentes.\newline

 Se ha comenzado con el prototipo que se ha utilizado para desarrollar el código de la aplicación, que figura en la siguiente imagen.

\begin{figure}[H]
    \centering
    \includegraphics[width=1\textwidth]{imagenes/montaje/prototipo.jpg}
    \caption{Prototipo para el desarrollo de funciones. Elaboración propia}
\end{figure}

 Tras extraer todos los componentes de dicho prototipo, se ha realizado la soldadura de componentes para las luces y los termistores, añadiendo un recubrimiento termoretráctil para cubrir los puntos de soldadura y cables que quedasen al aire. Esta fase se ha alargado más de lo necesario debido a problemas con los componentes, pruebas con el soldador y fallos de cálculo en cantidades de materiales necesarios, pero finalmente se ha conseguido un resultado aceptable. 

 \begin{figure}[H]
    \centering
    \includegraphics[width=1\textwidth]{imagenes/montaje/luces.jpg}
    \caption{Construcción de los circuitos de iluminación y temperatura. Elaboración propia}
\end{figure}



Una vez realizadas las soldaduras, se ha intentado acomodar todo dentro del coche que se escogió durante la planificación, de escala 1/24, pero aunque las mediciones realizadas parecían demostrar que era suficiente espacio, no se tuvo en cuenta la cantidad de cables del circuito. Por tanto, y viendo que no había posibilidad de implementar el circuito dentro de la maqueta, se buscó un coche escala 1/18, con el consiguiente tiempo de espera y retraso en la producción. 

Cuando llegó el siguiente modelo se desmontó entero, dejando únicamente la carrocería, la base de la maqueta y las ruedas. Una vez vacío el coche, se ubicó y reforzó el cableado mediante silicona caliente, fijándolo a la carrocería. Se añadieron también los termistores ubicados cerca de su lugar final, así como también se añadieron los dos motores que se indicaron en la planificación. 

Tras realizar las pruebas para comprobar que todos los sistemas funcionasen, surgieron problemas con los dos motores, pues los ejes no estaban adaptados a las ruedas que se iban a utilizar (las de la maqueta), eran demasiado grandes para el modelo, y además requerían de demasiada potencia para comenzar a funcionar debido a su baja eficiencia. Vistos estos problemas, se realizó la compra de unos motores más pequeños y eficientes, normalmente utilizados para modelismo y vehículos radiocontrol de media-alta gama. Estos motores mostraron buen rendimiento y un valor de voltaje muy inferior al requerido por los anteriores dejando mucho más espacio disponible en la parte frontal del vehículo, por lo que se implementaron en el circuito. 



Ya con todos los componentes instalados en la maqueta, se procedió a intentar integrar el arduino para un aspecto más limpio, añadiendo una caja de plástico debajo de la maqueta para facilitar la manipulación de los pines y permitir que la alimentación llegase a la placa, cosa que, de haber estado esta dentro de la maqueta, hubiera sido imposible. 

\begin{figure}[H]
    \centering
    \includegraphics[width=0.6\textwidth]{imagenes/montaje/ensamblado.jpg}
    \caption{Parte inferior de la maqueta, ya con la caja de plástico integrada. Elaboración propia}
\end{figure}

El último paso fue añadir unos soportes para que las ruedas no tocasen el suelo, e intentar dejar los pesos de la zona delantera y trasera lo más equilibrados posibles. 
\begin{figure}[H]
    \centering
    \includegraphics[width=1\textwidth]{imagenes/montaje/maqueta_final.jpg}
    \caption{Aspecto final de la maqueta mientras están activas las luces cortas e intermitente. Elaboración propia}
\end{figure}



Terminada esta tarea, queda completa la implementación del proyecto, aunque hay algunas funciones que no han podido implementarse que se detallarán a continuación.

\subsection*{Medidor de voltaje en tiempo real}

Si bien esta función parecía sencilla en un principio, el hecho de usar baterías para medir el voltaje que se le proporciona a los motores generaba un problema de compatibilidad con el módulo de medición de voltaje y la placa L928N. Un medidor de voltaje hubiera incluido demasiado trabajo y una inversión en diferentes componentes que aumentaban el coste del proyecto y sobrepasaban el presupuesto que se había pensado en un principio, que ya se había superado al tener que comprar un coche de mayor tamaño. 

Este problema produjo una cascada de funciones que no pudieron desarrollarse, la mayoría relacionadas con la batería.

\subsection*{Carga, autonomía y voltaje de la batería}

Las funciones relacionadas con el estado de las baterías requerían de un medidor de voltaje y otro de corriente, pues no existe manera aproximada de calcular el voltaje que se provee a los motores, necesario para obtener el porcentaje de batería restante ni la autonomía. Se ha intentado realizar otros métodos para implementar estas funciones sin éxito, por lo que, debido al tiempo restante en la planificación, se ha preferido eliminar estas funciones del proyecto. 

\subsection*{Tracción total}

Aunque en un principio se quiso implementar la tracción en ambos ejes, al realizar el montaje e incluir los componentes se han encontrado problemas a la hora de utilizar los motores que se indicaron en la planificación y, al cambiar a los otros motores, no se ha podido implementar por no tener doble eje, por lo que se ha preferido implementar únicamente tracción delantera. 

	% Trabajos futuros

	\chapter{Conclusiones y trabajos futuros}
\fancyhead[R]{8. Conclusiones}


\noindent\fbox{
	\parbox{\textwidth}{
    En este capítulo se tratarán las conclusiones que se han obtenido tras el trabajo, además de detallar las funciones que no se han podido implementar y/o se quieren realizar en un futuro, fuera del contexto del trabajo de fin de grado. También se tratarán los conocimientos y competencias adquiridas durante el desarrollo. 
	}
}


\section{Valoración general del proyecto}

Durante esta sección se hablará de los puntos importantes que se han observado durante la realización de este proyecto, así como los sucesos que han influido en él a la hora de desarrollarlo y las consecuencias que han tenido en tiempos, objetivos y resultados. 

\subsection{¿Qué ha funcionado?}

En general, si bien el desarrollo ha sido un tanto caótico por falta de experiencia en proyectos de esta envergadura, hay multitud de cosas que han funcionado como deberían. Entre ellas a destacar:

\begin{itemize}
    \item \textbf{Investigación y documentación}: El método utilizado para recopilar la información necesaria para este proyecto, iniciando con la búsqueda de conceptos clave y su significado para construir una base, continuando con la lectura y realización de apuntes y anotaciones de términos, técnicas y datos importantes obtenidos de fuentes citadas, y finalizando con la utilización de dicha información para construir las secciones teóricas o prácticas del proyecto, ha sido totalmente satisfactorio. Se ha sentado una base de conocimientos suficientemente útil para no tener que consultar muchas más fuentes tras la investigación inicial, exceptuando la búsqueda de imágenes. 
    \item \textbf{Programación con FreeRTOS y Processing}: La extensa documentación de ambos ha facilitado mucho el desarrollo, además del apoyo de multitud de webs, foros y papers que han utilizado estas tecnologías. En el caso de Processing, el uso de ControlP5 ha agilizado el proyecto por su documentación y ejemplos mostrados en su web, haciendo el desarrollo de la interfaz el doble de rápido de lo que se esperaba en un principio.
\end{itemize}

\subsection{¿Qué errores se han cometido?}

Existen multitud de decisiones que fueron erróneas en un principio, teniendo que corregirlas a posteriori. Aunque la mayoría quedaron solucionados, se han dejado sin cumplir varios objetivos por una mala planificación. Algunos de estos problemas son los siguientes:

\begin{itemize}
    \item \textbf{Componentes}: Al realizar la planificación se intentó tener en cuenta todos los factores que determinaban cuál era el componente correcto, pero algunas variables causaron que se tuviera que comprar otras piezas diferentes, como es el caso de los motores o la maqueta. 
    \item \textbf{Funciones que no se pudieron implementar}: Las funciones relacionadas con el control de la batería, ya nombradas en otros capítulos anteriores, no pudieron realizarse por falta de documentación en ese campo, además de haber subestimado la dificultad y el coste que conllevaban. 
    \item \textbf{Gestión de tiempo}: Aunque se realizó una temporización que, a priori, parecía correcta, los errores y problemas en las diferentes fases de desarrollo han hecho que los plazos no se cumplan en varias secciones, como indica el gráfico posterior en comparación a la figura (pend), mostrada en el capítulo de temporización. Por esta razón se tuvieron que descartar algunas mejoras en la interfaz y el circuito aunque, exceptuando el mayor tiempo de desarrollo, tampoco influyó demasiado en la calidad final del trabajo. 
\end{itemize}


\begin{figure}[h]
    \centering
    \includegraphics[width=1\textwidth]{imagenes/Gantt_final.png}
    \caption{Diagrama de Gantt con las tareas y su tiempo real de finalización. Elaboración propia en GanttProject}
\end{figure}

\subsection{Objetivos en retrospectiva}

A continuación se detallarán los objetivos que se han cumplido durante el desarrollo, así como aquellos que no se han podido alcanzar debido a las razones comentadas anteriormente. 


\subsubsection*{Objetivos de Investigación}

\begin{itemize}
    \item \textcolor{teal}{\faCheck} \textbf{O-I.1} - Comprender el funcionamiento de una centralita a nivel básico.
    \item \textcolor{teal}{\faCheck}  \textbf{O-I.2} - Sintetizar un conjunto de funciones que podrían ser replicadas en el proyecto.
    \item \textcolor{teal}{\faCheck}  \textbf{O-I.3} - Analizar las alternativas a la hora de implementar los diversos módulos.
    \item \textcolor{teal}{\faCheck}  \textbf{O-I.4} - Conocer los diferentes sistemas del vehículo y cómo interactúan con el entorno mediante sensores y actuadores.
    \item \textcolor{teal}{\faCheck}  \textbf{O-I.5} - Investigar sobre los sistemas de tiempo real y sus limitaciones.
    \item \textcolor{teal}{\faCheck}  \textbf{O-I.6} - Entender las directrices del código propietario y buscar alternativas de código libre para implementar el proyecto.
    \item \textcolor{teal}{\faCheck} \textbf{O-I.7} - Analizar las licencias que se le podrían atribuir al proyecto y escoger una de estas acorde a las características.
    \item \textcolor{teal}{\faCheck} \textbf{O-I.8} - Conocer el impacto de las ECU en el panorama automovilístico actual. 
\end{itemize}

\subsubsection*{Objetivos de Diseño}
\begin{itemize}
    \item \textcolor{teal}{\faCheck}  \textbf{O-D.1} - Diseñar actores y casos de uso para encontrar los límites y soluciones de los que precise el proyecto.
    \item \textcolor{red}{\faRemove}  \textbf{O-D.2} - Diseñar un esquemático para representar cómo interactúan los diferentes módulos con la ECU, así como mostrar los datos que se transmitan.
    \item \textcolor{teal}{\faCheck}  \textbf{O-D.3} - Programar la ECU para la recepción, tratamiento y envío de datos.
    \item \textcolor{teal}{\faCheck}  \textbf{O-D.4} - Implementar una interfaz de usuario para poder mostrar los módulos que buscamos.
    \item \textcolor{teal}{\faCheck}  \textbf{O-D.5} - Recrear este sistema en una maqueta de manera física si la temporización lo permitiera.
\end{itemize}

\section{Trabajos futuros}

Una vez finalizado el trabajo, y ya como desarrollo personal o por hobby, se intentarán implementar aquellas funciones que tienen que ver con la batería, además de perfeccionar el proyecto y añadir otras funcionalidades que puedan ser interesantes. También se trabajará en hacer el módulo inalámbrico, únicamente alimentado por baterías, con una comunicación bluetooth. 

\section{Competencias y conocimientos adquiridos}

Con la realización de este trabajo se han adquirido conocimientos y habilidades en multitud de campos, además de consolidar aquellos obtenidos durante la carrera en múltiples asignaturas. Si bien el grueso de lo aprendido pertenece al área del hardware y la ingeniería de computadores, con la construcción del circuito, elección de componentes, y programación de un sistema empotrado con un RTOS, también se han utilizado conocimientos de otros campos estudiados. Algunos de estos son relativos a la rama de Ingeniería del Software, como puede ser la creación de la interfaz de usuario e ingeniería de requisitos, así como también se ha utilizado la base de electrónica para calcular voltajes, resistencias y estructura del circuito en general. 

Ha sido un primer contacto muy educativo con los sistemas de tiempo real y las interfaces de usuario, temas que, si bien se habían comentado de manera superficial en algunas asignaturas cursadas, nunca se habían profundizado al nivel que se ha alcanzado en este trabajo. También el manejo del tiempo, la planificación y la constancia han hecho que, de ahora en adelante, sea más sencillo iniciar un proyecto a este nivel, ya sea por simple interés o \textit{hobby}, o por necesidad en el futuro laboral. 

\section{Experiencia personal}

Después de todos estos meses trabajando en el proyecto, con sus altos y sus bajos, no puedo estar mucho más contenta de lo que estoy con lo que se ha conseguido. Al comienzo fue difícil, pues no sabía cómo empezar a escribir, cómo organizar todo el trabajo, ni cuánto tiempo iba a llevar. Han pasado ya unos meses desde que le presenté la idea al tutor, y desde ahí en adelante cada vez se ha hecho más sencillo abordar las distintas secciones que conforman el trabajo. Ha sido una experiencia gratificante y agobiante a partes iguales, pero me quedo con todo lo que he aprendido y mejorado en todos los sentidos. 


	
	\newpage
	\bibliography{bibliografia}
	\bibliographystyle{plain}
	
\end{document}

