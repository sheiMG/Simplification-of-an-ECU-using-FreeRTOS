\thispagestyle{empty}

\begin{center}
{\large\bfseries Control de elementos de un automóvil \\ mediante un Sistema Empotrado }\\
\end{center}
\begin{center}
Sheila Martínez Gómez\\
\end{center}

%\vspace{0.7cm}

\vspace{0.5cm}
\noindent\textbf{Palabras clave}: \textit{software libre, sistema empotrado, ECU, microcontrolador, centralita, RTOS, automoción}
\vspace{0.7cm}

\noindent\textbf{Resumen}\\

El grueso de este Trabajo de Fin de Grado consiste en la programación de un sistema empotrado para controlar diversas funciones que realiza un vehículo, utilizando un RTOS (un sistema operativo de tiempo real) para garantizar la respuesta acotada a las señales enviadas por el usuario.\newline

La realización del trabajo comienza con una investigación acerca de los diferentes subsistemas que conforman un vehículo, así como su evolución e impacto en la sociedad actual. Una vez adquirida la base de conocimientos necesarios, se procede a realizar la programación de la centralita basada en el Sistema Empotrado, además de una interfaz de usuario que permita iniciar y detener las tareas asociadas a las funciones del automóvil.\newline

Finalmente se construye una maqueta de un vehículo mostrando los sistemas desarrollados durante el proyecto en funcionamiento.\newline

\cleardoublepage

\begin{center}
	{\large\bfseries Same, but in English}\\
\end{center}
\begin{center}
	Student's name\\
\end{center}
\vspace{0.5cm}
\noindent\textbf{Keywords}: \textit{open source, embedded system, ECU, microcontroller, RTOS, automotive}, \textit{floss}
\vspace{0.7cm}

\noindent\textbf{Abstract}\\

The aim of this Final Degree Project is the programming of an embedded system to control various functions performed by a vehicle, using a RTOS (real-time operating system) to guarantee a time-delimited response to the signals provided by the user.\newline

The work starts with an investigation of the different subsystems that make up a vehicle, as well as their evolution and impact on today's society. Once the required knowledge base has been acquired, the programming of the ECU (Electronic Control Unit) will be carried out, as well as the user interface to start and stop the task associated with the car's functions.\newline

Finally, a car model is built to show the functioning of the developed subsystems during the project.\newline

\cleardoublepage

\thispagestyle{empty}

\noindent\rule[-1ex]{\textwidth}{2pt}\\[4.5ex]

D. \textbf{Jesús González Peñalver}, Profesor(a) del Departamento de \textbf{Ingeniería de Computadores, Automática y Robótica}

\vspace{0.5cm}

\textbf{Informo:}

\vspace{0.5cm}

Que el presente trabajo, titulado \textit{\textbf{Control de elementos de un automóvil mediante un Sistema Empotrado}}, ha sido realizado bajo mi supervisión por \textbf{Sheila Martínez Gómez}, y autorizo la defensa de dicho trabajo ante el tribunal que corresponda.

\vspace{0.5cm}

Y para que conste, expiden y firman el presente informe en Granada a Junio de 2018.

\vspace{1cm}

\textbf{El/la director(a)/es: }

\vspace{5cm}

\noindent \textbf{Jesús González Peñalver}

\chapter*{Agradecimientos}

A mis padres y mi familia, por apoyarme aun estando tan lejos de casa, dándome su ayuda y sus ánimos en los momentos más duros de la carrera para seguir adelante.\newline

A Kike, por estar siempre a mi lado para sacarme una sonrisa, ayudarme con todo lo que está en su mano, y hacer que los días sean más bonitos y llevaderos.\newline

A todos los amigos de la carrera, y los que están lejos, a Aroa, por tener siempre alguna frase perfecta para el momento correcto, a Álex, por su comprensión y su humor, a David, por todas esas tardes en la universidad matando el tiempo (y estudiando cuando tocaba).\newline

Y a mi tutor, Jesús, por la comunicación necesaria y los consejos para llevar este trabajo de la mejor manera posible.\newline


